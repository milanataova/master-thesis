%%%%%%% Document settings %%%%%%%%%%
\documentclass{DESSThesis}

%%%%%%%%%%% Additional Packages %%%%%%%%%%%%%%%%%%%%%%%%%%%
\usepackage{csquotes}
\usepackage{microtype}
\usepackage{hyperref}
\hypersetup{
  colorlinks=true,
  linkcolor=black,
  citecolor=black,
  urlcolor=black
}

\usepackage{booktabs}      % Professional quality tables
\usepackage{multirow}      % Tables with merged rows
\usepackage{longtable}     % Tables spanning multiple pages

%%%%%%%%%%% Personalized commands and definitions %%%%%%%%%
% Add custom commands if needed

% Table of contents depth: 1 = sections only (no subsections)
\setcounter{tocdepth}{1}

%%%%%%%%%%%%%%%%%%%%%% Hyphenations %%%%%%%%%%%%%%
\hyphenation{trac-ta-ble}
\hyphenation{Was-ser-stein}
\hyphenation{Kol-mo-go-rov-Smir-nov}
\hyphenation{syn-the-tic}

%%%%%%%%%%%%%%%%%%%%%%%%%%%%%%%%%%%%%%%%%%%%%%%%%%

\graphicspath{{img/}{figures/}}
\addbibresource{references/references.bib}

\begin{document}

%%%%%%%%%%%%%%%%%%%%% Frontpage %%%%%%%%%%%%%%%%%%%%%%%%%%%%%%%
\def \TypeofThesis{Master's Thesis}
\def \TitleofThesis{Comparing Life Satisfaction Scales for Synthetic Respondents: A Large Language Model Approach}
\def \AuthorofThesis{Milana Taova}
\def \Supervisor{Prof.\ Dr.\ Markus Strohmaier}
\def \Advisor{Georg Ahnert}

%%%%%%%%%%%%%%%%%%%%%%%%%%%%%%%%%%%%%%%%%
% Academic Title Page
% LaTeX Template
% Version 2.0 (17/7/17)
%
% This template was downloaded from:
% http://www.LaTeXTemplates.com
%
% Original author:
% WikiBooks (LaTeX - Title Creation) with modifications by:
% Vel (vel@latextemplates.com)
% Lorena Reintgen
%
% License:
% CC BY-NC-SA 3.0 (http://creativecommons.org/licenses/by-nc-sa/3.0/)
% 
%
%%%%%%%%%%%%%%%%%%%%%%%%%%%%%%%%%%%%%%%%%

%----------------------------------------------------------------------------------------
% TITLE PAGE
%----------------------------------------------------------------------------------------

\begin{titlepage} % Suppresses displaying the page number on the title page and the subsequent page counts as page 1
\newcommand{\HRule}{\rule{\linewidth}{0.5mm}} % Defines a new command for horizontal lines, change thickness here
\center % Centre everything on the page
%\renewcommand{\baselinestretch}{1.2} %vertical spacing between lines
%------------------------------------------------
%  Headings
%------------------------------------------------
\textsc{\LARGE \TypeofThesis}\\[1.5cm] % Main heading such as the name of your university/college
%------------------------------------------------
%  Title
%------------------------------------------------
\HRule\\[0.4cm]
\linespread{1.7}\selectfont
{\huge\bfseries \TitleofThesis}\\ % Title of your document
\HRule\\[1.5cm]
\linespread{1.2}\selectfont
%------------------------------------------------
%  Author(s)
%------------------------------------------------
    \vfill
    \textit{\large submitted by}\\[0.5cm] % Major heading
\textsc{\Large \AuthorofThesis}\\[0.5cm] % Minor heading
    \vfill

{\large\textit{Submitted to the}}\\
\Large Chair of Data Science in the Economic and Social Sciences \\
    {\large\textit{within the}}\\
    Faculty of Business Administration \\
    at the University of Mannheim
    
    %------------------------------------------------
%  Date
%------------------------------------------------
\vfill\vfill\vfill % Position the date 3/4 down the remaining page
{\large\today} % Date, change the \today to a set date if you want to be precise

\vfill\vfill\vfill
{ \large \center 
\textit{Advisor:}\\
\Advisor
}

\vfill
{ \large \center 
\textit{Supervisor:}\\
\Supervisor
}


%------------------------------------------------
%  Logo
%------------------------------------------------
%\vfill\vfill
%\includegraphics[width=0.2\textwidth]{placeholder.jpg}\\[1cm] % Include a department/university logo - this will require the graphicx package
 
%----------------------------------------------------------------------------------------
\vfill % Push the date up 1/4 of the remaining page
\end{titlepage}

%\vfill
\cleardoublepage

\tableofcontents

%%%%%%%%%%%%%%%%%%%%% Abstract %%%%%%%%%%%%%%%%%%%%%%%%%%%%%
\cleardoublepage

\thispagestyle{plain}
\begin{abstract}
Large Language Models (LLMs) have been proposed as ``silicon samples'' capable of generating synthetic survey responses. With many measurement instruments available for constructs such as life satisfaction, it remains unclear whether some yield better results in silicon sampling than others. I investigate to what extent LLMs can approximate human life satisfaction distributions from the World Values Survey (WVS) Wave 7, how questionnaire design affects approximation quality, and whether ensemble approaches can improve alignment. I used three LLMs (LLaMA 3.1 8B, LLaMA 3.3 70B, Qwen 2.5 72B) to generate synthetic responses across five scales for demographically-segmented populations in four countries.

Results reveal moderate approximation quality overall, with substantial variation across conditions. Questionnaire design significantly affects performance: the Original WVS and Satisfaction with Life Scale yield the best approximations, while the Reverse Scale performs considerably worse. Contrary to expectations, the smaller model outperforms larger ones. Variance decomposition shows that health status and model choice are the strongest predictors of approximation quality, while income shows no significant effect. Ensemble approaches combining predictions across questionnaires improve alignment by approximately 26\% over the best individual scale. These findings provide guidance for researchers considering LLM-generated survey data and contribute simulated evidence to the jingle-jangle problem in psychological measurement.
\end{abstract}
\cleardoublepage

%%%%%%%%%%%%%%%%%%%%% Chapters %%%%%%%%%%%%%%%%%%%%%%

\pagenumbering{arabic}  % Switch to Arabic numerals for main content

% Chapter 1: Introduction

\chapter{Introduction}
\label{ch:introduction}

Large-scale surveys remain the gold standard for understanding subjective well-being across populations, yet they come with substantial costs. The World Values Survey, for instance, requires years of coordination across dozens of countries and significant financial resources \parencite{haerpfer2022wvs}. These constraints limit researchers' ability to study specific populations, pilot-test survey instruments, or augment existing datasets. Large language models (LLMs) offer a potential alternative: generating synthetic survey responses that approximate human response patterns without the overhead of traditional data collection \parencite{anthis2025position}.

The measurement of life satisfaction presents a particularly interesting test case. Multiple established instruments, including single-item direct assessments, metaphorical scales like the Cantril Ladder \parencite{cantril1966pattern}, and multi-item inventories such as the Satisfaction with Life Scale \parencite{diener1985swls} and Oxford Happiness Questionnaire \parencite{hills2002ohq}, all purport to measure the same underlying construct, yet they differ substantially in their format, cognitive demands, and response patterns. LLM outputs are strongly affected by prompt perturbations \parencite{tjuatja2024llms}, but unlike human respondents, LLMs can be prompted with many different scales without fatigue effects. This creates an opportunity to systematically compare how different measurement approaches perform in silicon sampling.

In this thesis, I investigate whether LLMs can generate synthetic survey responses that approximate real human life satisfaction distributions from the World Values Survey Wave 7. The evaluation is structured around three research questions:

\begin{description}
    \item[RQ1:] To what extent can LLMs approximate human responses in life satisfaction surveys?
    \item[RQ2:] Are certain types of life satisfaction scales better approximated by LLMs than others?
    \item[RQ3:] Can alignment with human survey responses be improved by combining results across multiple scales?
\end{description}

To address these questions, this thesis employs a systematic methodological framework in which multiple LLMs generate responses to different life satisfaction questionnaires across diverse countries and demographic segments. Distributional alignment is quantified using Wasserstein distance, with variance decomposition identifying which factors most influence approximation quality. The contributions are twofold: \textit{methodologically}, this work demonstrates that combining results from multiple scales can improve alignment with human response distributions; \textit{empirically}, it provides systematic evidence revealing that the choice of a response scale and respondent health status are the primary determinants of approximation quality, that smaller models can outperform larger ones, and that ensemble approaches substantially improve alignment.

The remainder of this thesis is organized as follows. Chapter~\ref{ch:background} provides background on life satisfaction measurement and LLM capabilities. Chapter~\ref{ch:related-work} reviews prior work on synthetic survey data generation. Chapter~\ref{ch:method} details the methodology, including data sources, life satisfaction scales, and analytical approaches. Chapter~\ref{ch:results} presents the empirical findings. Chapter~\ref{ch:limitations} discusses limitations, and Chapter~\ref{ch:conclusion} concludes with implications and directions for future research.

% Chapter 2: Background
% This chapter provides foundational concepts for understanding life satisfaction measurement,
% large language models as survey respondents, and the emerging paradigm of synthetic survey data

\chapter{Background}
\label{ch:background}

This chapter establishes the conceptual foundations for the thesis. Section~\ref{sec:life-satisfaction-measurement} defines life satisfaction as a psychological construct, explains its importance for research and policy, and describes the principal measurement approaches. Section~\ref{sec:llms-survey-respondents} explains how large language models can be employed to simulate human survey responses. Section~\ref{sec:synthetic-data-promise-challenge} discusses the potential benefits and validity challenges of synthetic survey data, identifying the specific gaps this thesis addresses.

%-----------------------------------------------------------
\section{Life Satisfaction Measurement}
\label{sec:life-satisfaction-measurement}

\subsection{Defining Life Satisfaction}
\label{subsec:defining-life-satisfaction}

Life satisfaction refers to an individual's global cognitive evaluation of their life as a whole \parencite{diener1985swls}. Unlike momentary emotional states or domain-specific assessments, life satisfaction captures a reflective judgment: when people consider their lives in totality, how favorably do they evaluate them? This cognitive-evaluative component distinguishes life satisfaction from related concepts such as positive affect (feeling good) or eudaimonic well-being (finding meaning and purpose).

The construct sits within the broader framework of subjective well-being, which encompasses both cognitive evaluations (life satisfaction) and affective experiences (positive and negative emotions). While these components are correlated, they represent distinct psychological phenomena. A person might report high life satisfaction despite experiencing frequent negative emotions, or vice versa. This thesis focuses specifically on the cognitive-evaluative component, life satisfaction, because it is the most commonly measured aspect of subjective well-being in large-scale cross-national surveys.

Life satisfaction is inherently subjective: it reflects how individuals perceive and evaluate their own lives rather than external assessments of their circumstances. This subjectivity is both a strength and a limitation. It captures what objective indicators cannot (whether material conditions translate into lives people find worthwhile), but it also means that responses are shaped by expectations, social comparisons, cultural norms, and the specific way questions are framed.

\subsection{Why Life Satisfaction Matters}
\label{subsec:why-life-satisfaction}

Life satisfaction has become a focal point for research and policy because traditional development indicators provide an incomplete picture of human welfare. Gross domestic product, life expectancy, and educational attainment describe material conditions but say little about whether those conditions translate into lives people experience as worthwhile. This limitation has driven growing interest in subjective well-being as a complement to traditional metrics \parencite{undp2022hdr}.

The policy relevance of life satisfaction data is substantial. Governments increasingly incorporate well-being measures into national statistics and policy evaluation; the United Kingdom, for example, regularly tracks national well-being as part of official housing and social surveys \parencite{ukgov2025wellbeing}. Life satisfaction surveys inform policies ranging from urban planning to healthcare resource allocation.

Cross-national surveys have documented patterns invisible to objective measurement. The World Values Survey, spanning dozens of countries over multiple decades, has revealed that wealth and well-being diverge substantially beyond a certain threshold; that health status shapes life evaluations more strongly than income in many contexts; and that cultural factors produce systematic response differences even under similar objective conditions \parencite{haerpfer2022wvs}. These insights depend entirely on asking people directly---there is no objective measurement that can substitute for self-report.

\subsection{Measurement Approaches}
\label{subsec:measurement-approaches}

Life satisfaction can be measured through various approaches that differ in their format, cognitive demands, and theoretical assumptions. Understanding these differences is essential for interpreting response patterns and evaluating whether synthetic respondents can replicate them. Notably, measuring life satisfaction using scales with many items or multiple scales at once creates substantial effort for human participants and likely reduces data quality as respondents become less careful in their responses. This is a limitation that does not apply to LLMs, which can be administered any number of scales without fatigue effects.

\textbf{Single-item direct assessment.} The most straightforward approach asks respondents to rate their life satisfaction on a numerical scale. The World Values Survey question---\textit{``All things considered, how satisfied are you with your life as a whole these days?''}---exemplifies this approach, using a 1--10 scale anchored by ``completely dissatisfied'' and ``completely satisfied.'' Single-item measures are economical and face-valid, though they sacrifice the reliability benefits of multiple items \parencite{wanous1997overall}.

\textbf{Metaphorical framing.} The Cantril Self-Anchoring Scale \parencite{cantril1966pattern} presents life satisfaction through the metaphor of a ladder, with the top representing the best possible life and the bottom the worst. Respondents indicate which step represents their current standing. This approach may engage different cognitive processes than direct numerical rating, as respondents must first construct personal definitions of best and worst possible lives before locating themselves on the scale.

\textbf{Multi-item inventories.} Instruments like the Satisfaction with Life Scale (SWLS) \parencite{diener1985swls} use multiple items to assess the construct. The five-item SWLS asks respondents to rate agreement with statements such as \textit{``In most ways my life is close to my ideal''} and \textit{``If I could live my life over, I would change almost nothing.''} Multiple items improve reliability through aggregation and can capture different facets of overall life evaluation.

\textbf{Comprehensive well-being scales.} Broader instruments like the Oxford Happiness Questionnaire (OHQ) \parencite{hills2002ohq} assess life satisfaction alongside related constructs such as self-esteem, sense of purpose, and social relationships. The 29-item OHQ provides a more comprehensive assessment but requires substantially more respondent time and may conflate distinct aspects of well-being.

These measurement approaches are not interchangeable. Empirical research has shown that different formats elicit systematically different response patterns in human respondents, even when ostensibly measuring the same construct \parencite{schwarz2001asking}. Whether a single-item WVS question and a multi-item SWLS capture equivalent information---or engage different cognitive processes that produce different results---remains an open question with implications for research comparability and data synthesis. This variation across instruments in human data makes life satisfaction measurement a useful test case for evaluating whether LLMs exhibit similar sensitivity to questionnaire format.

\subsection{Measurement Challenges}
\label{subsec:measurement-challenges}

Several challenges complicate life satisfaction measurement. First, response patterns are sensitive to question wording, scale format, and context effects. The order in which questions appear, the specific anchors used, and even transient mood states can influence responses \parencite{schwarz1999self}.

Second, cultural factors shape how people interpret and respond to satisfaction questions. Response styles vary across cultures: some populations show stronger tendencies toward extreme responses, while others cluster around scale midpoints \parencite{roberts2016response}. These patterns may reflect genuine differences in satisfaction levels, cultural norms about expressing satisfaction, or differential interpretation of scale points.

Third, the field faces what \textcite{elson2023measures} describe as a proliferation problem: too many instruments measuring the same construct. With numerous scales available---each with proponents, validation studies, and accumulated research---selecting among them is difficult, and synthesizing findings across studies using different measures poses challenges. \textcite{wulff2025jinglejangle} frame this as the ``jingle-jangle'' problem, where instruments with the same label may measure different things (jingle fallacy) or instruments with different labels may measure the same thing (jangle fallacy).

These challenges are not merely academic. They affect the validity of cross-national comparisons, the interpretation of trends over time, and the practical utility of life satisfaction data for policy decisions. They also raise questions about whether synthetic respondents, generated by language models trained on human text, can navigate these measurement complexities in ways that approximate human response patterns.

%-----------------------------------------------------------
\section{Large Language Models as Survey Respondents}
\label{sec:llms-survey-respondents}

\subsection{Overview of Large Language Models}
\label{subsec:llm-overview}

Large language models (LLMs) are neural network systems trained on extensive text corpora to predict and generate human-like text. Modern LLMs, built on transformer architectures with billions of parameters, have demonstrated remarkable capabilities in natural language understanding, reasoning, and generation \parencite{llama3herd2024, qwen25report2024}. These models learn statistical patterns from their training data---which includes books, articles, websites, and other text sources---enabling them to produce contextually appropriate responses to diverse prompts.

The scale of contemporary LLMs is substantial. Models such as LLaMA 3.3 (70 billion parameters) and Qwen 2.5 (72 billion parameters) are commonly used open-weights language models \parencite{llama3herd2024, qwen25report2024}. This scale enables nuanced understanding of context, including the ability to adopt specified personas or respond from particular demographic perspectives when appropriately prompted.

\subsection{The Silicon Sampling Paradigm}
\label{subsec:silicon-sampling}

\textcite{argyle2023silicon} introduced the concept of ``silicon sampling,'' using LLMs to generate synthetic survey responses that simulate human populations. The core insight is that LLMs, having been trained on vast amounts of human-generated text including survey responses, opinion pieces, and demographic-specific content, may have implicitly learned patterns of how different types of people respond to different types of questions.

In the silicon sampling paradigm, researchers provide LLMs with demographic information and then pose survey questions. The model generates responses that, ideally, reflect how a person with those characteristics might actually respond. If LLMs can accurately capture these demographic-response patterns, they could generate synthetic survey samples at a fraction of the cost and time required for traditional data collection.

This approach has generated both enthusiasm and skepticism. Proponents note that LLMs might capture response patterns that are difficult to model explicitly, potentially enabling rapid prototyping of survey instruments or supplementation of sparse demographic categories in existing datasets. Critics raise concerns about whether LLMs genuinely capture human response variation or merely reflect biases and stereotypes embedded in their training data \parencite{bisbee2024synthetic, gallegos2024bias}.

%-----------------------------------------------------------
\section{The Promise and Challenge of Synthetic Survey Data}
\label{sec:synthetic-data-promise-challenge}

\subsection{Why Synthetic Survey Data?}
\label{subsec:why-synthetic}

The appeal of LLM-generated survey data stems from a fundamental asymmetry: collecting real survey data is expensive and slow, while generating synthetic data is cheap and fast. Large cross-national surveys require years of coordination and substantial funding, whereas an LLM can generate thousands of responses in hours for minimal cost.

This asymmetry creates several potential applications. Researchers could pilot-test survey instruments before investing in actual data collection, identifying problematic questions or unexpected response patterns early in the research process. Synthetic data could also augment sparse cells in existing datasets---the demographic combinations that surveys inevitably undersample due to practical constraints---or provide baseline distributions for power analyses and simulation studies. Additionally, synthetic responses might support rapid iteration on survey design, allowing researchers to test multiple question formulations without repeated data collection.

Beyond cost savings, synthetic data offers accessibility benefits for researchers without large grants or institutional survey infrastructure, enabling preliminary investigations that would otherwise be infeasible. Studies of hard-to-reach populations might benefit from synthetic augmentation, and in contexts where privacy concerns limit data sharing, synthetic data could provide alternatives for secondary analysis.

However, these potential benefits rest on a critical assumption: that synthetic responses approximate real ones closely enough for the intended purpose. This assumption requires empirical validation.

\subsection{Challenges for Synthetic Survey Data}
\label{subsec:synthetic-challenges}

Several challenges complicate the use of LLM-generated survey data. The most fundamental concerns validity: do synthetic responses actually reflect how real people would respond? Most existing research has focused on whether LLMs can reproduce aggregate statistics---means, correlations, or regression coefficients---from real surveys \parencite{argyle2023silicon}. However, matching aggregate statistics does not guarantee that synthetic data faithfully represents the underlying population. For many research purposes, the full distribution of responses matters: not just average satisfaction levels, but how responses spread across the scale and whether the overall shape matches reality.

Generalization across constructs presents another challenge. While \textcite{argyle2023silicon} demonstrated that LLMs can capture demographic differences in political attitudes, it remains unclear whether this extends to subjective well-being constructs like life satisfaction. Political attitudes may be more explicitly represented in LLM training data through news articles, opinion pieces, and social media, whereas life satisfaction involves personal evaluation that may be less directly captured in text corpora.

The sensitivity to measurement format observed in human respondents also raises questions for synthetic data. If different questionnaire designs elicit different response patterns from humans, do LLMs show similar sensitivity? How approximation quality varies across scales---single-item questions, multi-item scales, metaphorical framings, or reversed scales---requires systematic investigation.

Demographic representation in LLM training data may introduce additional biases. Models trained predominantly on English-language text from Western contexts may perform differently when simulating respondents from other cultural backgrounds, and whether LLMs can accurately capture cross-national variation in life satisfaction, or whether they default to patterns dominant in their training data, remains an open question.

Finally, if individual approaches yield imperfect approximations, strategies for improvement are needed. Whether ensemble methods combining predictions from multiple questionnaires or models can improve alignment remains unexplored in the synthetic survey data literature.

%-----------------------------------------------------------
% End of Chapter 2

% Chapter 3: Related Work
% This chapter reviews prior research on LLMs as synthetic survey respondents,
% questionnaire design effects, and demographic conditioning approaches

\chapter{Related Work}
\label{ch:related-work}

This chapter reviews the empirical literature on LLMs as synthetic survey respondents, questionnaire design effects, and demographic conditioning.

%-----------------------------------------------------------
\section{LLMs as Synthetic Survey Respondents}
\label{sec:llm-synthetic-respondents}

The systematic use of LLMs as synthetic survey respondents was introduced by \textcite{argyle2023silicon}, who coined the term ``silicon sampling.'' Their study demonstrated that GPT-3, when provided with demographic backstories drawn from real survey respondents, could produce response distributions correlating with actual political attitude data from the American National Election Studies. The model reproduced known demographic patterns, such as higher Democratic identification among Black respondents and higher Republican identification among white evangelical Christians. This work established that language models trained on vast text corpora contain implicit demographic-attitude relationships that can be elicited through appropriate prompting.

Argyle et al.\ validated their approach primarily through correlation with aggregate survey statistics. However, correlation does not capture whether synthetic data reproduces the full shape of human distributions; two distributions can correlate highly while differing substantially in variance, skewness, or modality. Distributional metrics such as Wasserstein distance provide more comprehensive assessment by quantifying differences across the entire response range.

Subsequent research has explored the boundaries of the silicon sampling approach. \textcite{sarstedt2024silicon} provide guidelines for using silicon samples in consumer and marketing research, comparing GPT-generated responses to human responses across multiple constructs. They found that while LLMs show promise for preliminary tasks like survey pretesting and pilot studies, they are not reliable substitutes for human respondents in main studies intended to draw substantive conclusions. Their recommendation is to treat silicon samples as complements rather than replacements for human data collection.

\textcite{sun2024random} demonstrated that using only group-level demographic information---without individual backstories---language models can generate response distributions with correlations above 0.8 compared to actual U.S.\ polling data for many political questions. However, performance varied substantially by topic: questions about frequently polled attitudes showed stronger correspondence, while less commonly surveyed topics showed weaker alignment. This suggests that LLM performance depends on the availability of relevant patterns in training data---a consideration particularly important for subjective well-being constructs, which may be less explicitly represented in text corpora than political opinions. Indeed, the existing literature focuses predominantly on political attitudes; whether the silicon sampling paradigm extends to life satisfaction remains largely unexplored.

Cross-national extensions have revealed important limitations. \textcite{ma2024german} examined algorithmic fidelity in generating synthetic German public opinion, finding substantially reduced performance compared to American contexts. While the model captured some broad patterns, it failed to reproduce many Germany-specific relationships between demographics and opinions. This aligns with findings that American and English-language content predominates in LLM training data \parencite{li2024languageranker}. Similar performance variation might be expected across countries differing in language and cultural representation in training corpora.

The paradigm has attracted substantial criticism. \textcite{bisbee2024synthetic} provide a systematic critique, showing that while LLMs can match marginal distributions on many questions, they often fail to reproduce the correlational structure between variables, the patterns of which attitudes tend to co-occur within individuals. This matters because many social science analyses rely on relationships between variables rather than marginal distributions alone. They also documented troubling temporal instability: the same prompt yielded significantly different results over a three-month period, and minor wording changes produced substantial shifts in response distributions. These findings underscore the importance of systematic comparison across multiple measurement approaches.

\textcite{huang2025uncertainty} quantified these limitations by developing a framework for constructing confidence intervals when using LLM-simulated responses. Their analysis reveals that effective sample sizes are often below 100, and sometimes in single digits, even when nominal sample sizes are much larger. This occurs because LLM outputs lack the independence of human responses; the model's fixed parameters introduce correlations that reduce effective information content. When individual approaches yield imperfect approximations, combining information across multiple questionnaires or models might improve overall alignment---yet such ensemble strategies remain unexplored in the synthetic survey literature.

Despite these concerns, some evidence supports cautious optimism. In large-scale tests spanning 70 preregistered experiments, \textcite{hewitt2024predicting} found that GPT-4 predicted 91\% of variation in average treatment effects when adjusting for measurement error. However, they note that such simulations are best suited for exploratory research rather than definitive conclusions about human populations.

%-----------------------------------------------------------
\section{Questionnaire Design Effects in LLM Responses}
\label{sec:questionnaire-effects-llms}

Human survey respondents are sensitive to question wording, scale format, and context effects, phenomena extensively documented in survey methodology research. Whether LLMs exhibit similar sensitivities has important implications for their use as synthetic respondents.

\textcite{tjuatja2024llms} provide the most systematic examination of this question. They manipulated question order, response option ordering, scale direction, and other design features known to affect human responses, testing multiple models including GPT-3.5 and several open-source alternatives. Their findings reveal a complex picture: LLMs exhibit some human-like biases but not others, with patterns varying substantially across models and question types. For some design features, certain models showed sensitivity consistent with human patterns; for others, LLM responses were either insensitive to manipulations that affect humans or sensitive in inconsistent directions.

Most concerning, the authors found that LLMs can be highly sensitive to seemingly minor prompt variations, sometimes more sensitive than human respondents would be. Small phrasing changes that would be inconsequential for humans produced notable shifts in LLM outputs. This hypersensitivity raises questions about the robustness of LLM-generated survey data and suggests that comparing responses across distinct questionnaire formats measuring the same construct could reveal which measurement approaches produce consistent versus variable synthetic responses.

Research on human respondents documents that reversed items, where higher scores indicate lower construct levels, produce different psychometric properties than positively-worded items \parencite{suarez2018reversed, weijters2013reversed}. This ``reversed item bias'' reflects cognitive processing differences: respondents may fail to notice the reversal, apply different response strategies, or experience increased cognitive load when responding to negatively-worded items. The result is that scales containing reversed items sometimes show factor structures suggesting methodological artifacts rather than genuine construct dimensions.

Whether LLMs exhibit similar sensitivities to scale reversal remains underexplored. If LLMs struggle with reversed scales, either failing to recognize the reversal or responding inconsistently when scale direction changes, this would manifest as poorer approximation quality compared to standard formats. Similarly, metaphorical question framings such as the Cantril Ladder, which asks respondents to imagine their life position on a ladder from worst to best possible life, may elicit different processing than direct numerical ratings. Systematic comparison across these formats can provide empirical evidence on which measurement approaches LLMs handle well and which prove problematic. Yet studies typically employ a single questionnaire format, making it impossible to assess whether LLM performance varies with measurement approach.

This gap complements recent work by \textcite{wulff2025jinglejangle}, who addressed the jingle-jangle problem using semantic embeddings from LLMs. Their approach assesses whether instruments are semantically similar based on item content. While semantic analysis examines what instruments ask about, testing whether instruments elicit similar response patterns provides a behavioral rather than semantic comparison. Together, these approaches offer more comprehensive evidence about measurement relationships than either alone.

%-----------------------------------------------------------
\section{Demographic Conditioning and Persona Prompting}
\label{sec:demographic-conditioning}

How to condition LLMs on demographic information is a key methodological choice in generating synthetic survey responses. Researchers have developed various approaches, from simple attribute listing to elaborate narrative personas.

\textcite{lutz2025prompt} provide the most systematic evaluation, comparing prompting strategies across five open-source LLMs and 15 intersectional demographic groups. They examined direct role assignment (``You are a Black woman''), third-person framing, and interview-style formats where demographic information emerges through simulated dialogue. Interview-style prompts consistently outperformed other formats, producing responses with fewer stereotypical markers and better alignment with human data.

Surprisingly, \textcite{lutz2025prompt} found that smaller models sometimes outperformed larger ones on alignment metrics, challenging assumptions that larger models necessarily produce better synthetic survey data. This finding suggests that model selection for synthetic survey generation should be based on empirical validation rather than parameter count alone.

\textcite{hu2024persona} quantified the overall magnitude of persona effects, finding that persona variables account for less than 10\% of variance in model outputs for most tasks. This suggests that while demographic conditioning influences LLM responses, the effect may be smaller than commonly assumed. The optimal approach varies by demographic attribute: some characteristics (such as political orientation) are more reliably captured than others (such as income level). This variability implies that different demographic dimensions may show differential explanatory power in variance decomposition analyses.

However, persona-based approaches carry significant risks. \textcite{gupta2024bias} demonstrated that LLMs harbor deep-rooted biases that manifest when assigned personas. While models overtly reject stereotypes when asked directly (e.g., responding ``no'' to ``Are Black people less skilled at mathematics?''), they exhibit stereotypical presumptions when answering questions while adopting a persona. These biased assumptions appeared across 80\% of tested personas, with performance drops exceeding 70\% for some disadvantaged-group personas compared to majority-group personas. For subjective well-being measurement, this raises concerns that responses for low-income or poor-health personas may reflect stereotypical assumptions rather than actual patterns in human data.

\textcite{liu2024persona} extended this analysis to opinion generation, finding LLMs are 9.7\% less steerable toward incongruous personas---those combining statistically unusual trait combinations, such as political liberals supporting military spending. When prompted with such personas, LLMs sometimes generate stereotypical stances associated with one demographic trait rather than the target opinion. Incongruous combinations in life satisfaction contexts might include high-income individuals with poor health or low-income individuals reporting high satisfaction---cases where LLMs might default to stereotypical associations rather than capturing genuine human variation.

Cross-national applications face additional challenges. English-language content dominates LLM training corpora, with performance gaps of 5--15 percentage points common between English and other languages \parencite{li2024languageranker, gupta2025multilingual}. For survey simulation, this implies that cross-national comparisons may be confounded by differential model performance rather than genuine population differences. Comparing approximation quality across countries with different cultural contexts can help assess whether performance varies systematically with presumed training data representation.

In summary, the literature on LLMs as synthetic survey respondents reveals several gaps. First, most studies focus on political attitudes rather than subjective well-being constructs. Second, few studies systematically compare LLM performance across different questionnaire formats measuring the same construct. Third, ensemble approaches that combine responses across multiple instruments remain unexplored. This thesis addresses these gaps by comparing five life satisfaction questionnaires across multiple LLMs and countries, using distributional metrics to assess approximation quality and testing whether combining responses improves alignment with human data.

%-----------------------------------------------------------
% End of Chapter 3


% Chapter 4: Method
% This chapter describes the data sources, generation procedures, and analytical methods used to evaluate synthetic survey data quality

\chapter{Method}
\label{ch:method}

\section{Overview of Analytical Approach}
\label{sec:method-overview}

\subsection{Research Questions}
\label{subsec:research-questions}

This thesis evaluates the capacity of large language models (LLMs) to approximate human responses in life satisfaction surveys through a systematic comparison of synthetic and real-world survey data. The evaluation is structured around three research questions:

\textbf{Research Question 1 (RQ1):} \textit{To what extent can large language models approximate human responses in life satisfaction surveys?}

This foundational question examines whether LLM-generated responses exhibit distributional characteristics sufficiently similar to real human responses to be considered valid approximations, requiring quantitative metrics and empirically-grounded thresholds for approximation quality.

\textbf{Research Question 2 (RQ2):} \textit{Are certain types of life satisfaction scales or questionnaire designs better approximated by LLMs than others?}

This question investigates whether approximation quality varies systematically across different measurement instruments, including single-item assessments, metaphorical framings, and multi-item scales, with implications for optimal survey design in LLM-based simulation contexts.

\textbf{Research Question 3 (RQ3):} \textit{Can alignment with human survey responses be improved by combining simulation results across multiple scales?}

This question explores whether ensemble methods, combining predictions from multiple questionnaire variants, can yield superior approximations by reducing measurement error and capturing different facets of the underlying construct \cite{ghiselli1981measurement}.

\subsection{Methodology Pipeline}
\label{subsec:analytical-pipeline}

Figure~\ref{fig:methodology-pipeline} provides an overview of the methodology pipeline. The approach proceeds through five main stages:

\textbf{Real Survey Data.} World Values Survey Wave 7 provides the reference distributions of human life satisfaction responses. Four countries with non-normal distributions were selected to avoid spurious alignment (Section~\ref{sec:data-sources}).

\textbf{Data Preparation.} Survey responses are stratified into 36 demographic segments based on country (4), income level (3), and health status (3). This segmentation enables both aggregate country-level comparisons and fine-grained subgroup analysis (Section~\ref{subsec:wvs-data}).

\textbf{Prompt Design.} Each synthetic response is generated using a structured prompt combining three elements: a demographic persona describing the respondent's characteristics, a task instruction framing the survey context, and the questionnaire items themselves (Section~\ref{subsec:synthetic-data}).

\textbf{Synthetic Data Generation.} Three large language models generate responses to five life satisfaction scales, producing 16,200 synthetic responses (30 responses per segment $\times$ 36 segments $\times$ 15 model-questionnaire combinations). Multi-item scale responses undergo equipercentile equating to enable comparison on a common 1--10 metric (Sections~\ref{subsec:synthetic-data} and \ref{subsec:equipercentile-equating}).

\textbf{Distributional Comparison.} Synthetic and real distributions are compared using Wasserstein distance as the primary metric and Kolmogorov-Smirnov statistic as a secondary measure. These comparisons support three types of analysis: specification curve analysis across all analytical configurations, variance decomposition to quantify factor importance, and ensemble methods combining multiple scales (Section~\ref{sec:distance-metrics}).

\begin{figure}[htbp]
    \centering
    \includegraphics[width=\textwidth]{figures/fig_methodology_pipeline.png}
    \caption{Overview of the methodology pipeline. World Values Survey data from four countries undergoes data preparation to create 36 demographic segments (4 countries $\times$ 3 income levels $\times$ 3 health statuses). Prompt design combines persona, task, and questionnaire elements. Synthetic data generation uses three LLMs and five life satisfaction scales to produce 16,200 responses. Distributional comparison evaluates alignment between real and synthetic data using Wasserstein distance and KS statistic.}
    \label{fig:methodology-pipeline}
\end{figure}

%-----------------------------------------------------------
\section{Data Sources and Preparation}
\label{sec:data-sources}

\subsection{World Values Survey Wave 7}
\label{subsec:wvs-data}

The real-world survey data for this study comes from the World Values Survey (WVS) Wave 7, a large-scale international social survey conducted between 2017 and 2022 across numerous countries \cite{haerpfer2022wvs}. The WVS employs nationally representative probability sampling with stratification and clustering to capture population-level attitudes, values, and beliefs.

\subsubsection{Country Selection}

This study focuses on four countries: the United States (USA), Indonesia (IDN), the Netherlands (NLD), and Mexico (MEX). Country selection was guided by three criteria. First, geographic diversity ensures representation across continents (Americas, Europe, and Asia), capturing distinct cultural contexts. Second, variation in economic development levels---as measured by GDP per capita and Human Development Index \cite{undp2022hdr, worldbank2023wdi}---provides heterogeneity in socioeconomic conditions that may influence life satisfaction patterns.

Third, and most critically, countries were selected based on the distributional characteristics of their life satisfaction responses. The analysis deliberately focuses on countries exhibiting non-normal distributions---specifically, distributions that are skewed or multimodal rather than approximately Gaussian. This criterion is methodologically important because if both real and synthetic data follow normal distributions, good alignment could occur coincidentally due to the central limit theorem rather than reflecting meaningful approximation of human response patterns. By examining countries with more complex distributional shapes, the evaluation provides a more stringent test of LLM approximation quality.

Figure~\ref{fig:country-selection} illustrates this selection criterion. South Korea exhibits an approximately normal distribution, whereas Armenia and Mexico show pronounced skewness. Countries with normal distributions like South Korea were excluded to avoid spurious alignment that could confound the assessment of approximation quality.

\begin{figure}[htbp]
    \centering
    \includegraphics[width=\textwidth]{figures/fig_country_selection_distributions.png}
    \caption{Life satisfaction score distributions from World Values Survey Wave 7 for three countries examined during study design. Blue bars represent relative frequencies of responses on the 1--10 scale. South Korea exhibits an approximately normal distribution and was excluded from analysis. Armenia and Mexico show non-normal distributions---the types of skewed patterns sought for this study to avoid spurious alignment.}
    \label{fig:country-selection}
\end{figure}

\subsubsection{Life Satisfaction Measurement}

The primary outcome variable is life satisfaction, measured using WVS question Q49: \textit{``All things considered, how satisfied are you with your life as a whole these days?''} Respondents rate their satisfaction on a 10-point scale where 1 indicates ``completely dissatisfied'' and 10 indicates ``completely satisfied.'' This single-item measure provides a direct, face-valid assessment of overall life evaluation and serves as the reference distribution for all synthetic data comparisons.

\subsubsection{Demographic Variables}

Two demographic variables structure the subgroup analysis: income level and health status. These variables were selected based on Random Forest feature importance analysis, which identified them as the most predictive demographic factors for life satisfaction among available WVS variables, exceeding age, gender, education, marital status, and employment status.

Income level is derived from WVS question Q288, which asks respondents to place their household income on a 10-point scale. Responses were recoded into three categories using within-country terciles: Low (scores 1--3), Medium (scores 4--7), and High (scores 8--10). Within-country categorization ensures that each income category represents a comparable position within the national income distribution rather than absolute purchasing power.

Health status comes from question Q47, which asks respondents to rate their health on a five-point scale from very good to very poor. The original scale was collapsed into three levels to ensure adequate sample sizes: Good (very good + good), Fair, and Poor (poor + very poor). Figure~\ref{fig:feature-importance} displays the feature importance rankings. Further details on the segmentation structure are provided in Section~\ref{subsec:demographic-segments}.

\begin{figure}[htbp]
    \centering
    \includegraphics[width=0.8\textwidth]{figures/fig_feature_importance.png}
    \caption{Random Forest feature importance for predicting life satisfaction. Country, health status, and income level emerge as the most important predictors among WVS variables, justifying their use as segmentation factors for the subgroup analysis.}
    \label{fig:feature-importance}
\end{figure}

\subsubsection{Survey Weights}

All distributional comparisons incorporate WVS sampling weights (variable \texttt{W\_WEIGHT}). These weights are essential for producing nationally representative estimates given the complex survey design involving stratification and clustering. Weighted analyses ensure that the real WVS distributions reflect population-level life satisfaction patterns rather than being biased by differential sampling probabilities. Correspondingly, synthetic data receive matching weights (described in Section~\ref{subsec:synthetic-data}) to align their demographic composition with real WVS distributions, enabling valid comparisons.

%-----------------------------------------------------------
\section{Life Satisfaction Scales}
\label{sec:life-satisfaction-scales}

Five questionnaire variants were employed to measure life satisfaction, each operationalizing the construct through different approaches. This diversity allows examination of whether certain measurement formats are more amenable to LLM approximation (RQ2). The questionnaires vary along multiple dimensions: number of items (single vs. multi-item), response scale format (direct numeric vs. metaphorical framing), and cognitive complexity (standard vs. reversed scales).

\subsubsection{Original WVS: Standard WVS Question}

The Original WVS questionnaire replicates the standard WVS life satisfaction question used in the real survey data \parencite{haerpfer2022wvs}. Respondents answer: \textit{``All things considered, how satisfied are you with your life as a whole these days?''} using a 10-point scale where 1 indicates completely dissatisfied and 10 indicates completely satisfied. This single-item direct assessment serves as the primary benchmark, as synthetic responses can be compared directly to real WVS responses without scale transformation.

\subsubsection{Cantril Ladder Scale}

The Cantril Ladder questionnaire employs the Cantril Self-Anchoring Scale framing \cite{cantril1966pattern}, which uses a metaphorical representation of life quality. Respondents are asked to imagine a ladder with steps numbered from 0 at the bottom to 10 at the top, where the top represents the best possible life and the bottom represents the worst possible life. They then indicate which step of the ladder best represents their current life. Responses on the 0-10 scale are recoded to 1-10 for consistency with other questionnaires. This approach tests whether metaphorical framing affects LLM response patterns differently than direct assessment.

\subsubsection{Reverse Scale}

The Reverse Scale questionnaire presents the same question as Original WVS but with an inverted response scale: 1 indicates completely satisfied and 10 indicates completely dissatisfied. This design tests whether LLMs can correctly process reversed scale orientations, a cognitive task that sometimes challenges human respondents \cite{suarez2018reversed}. The Reverse Scale was also included to test LLM robustness to prompt variations, as initial pilot testing revealed that LLMs were generating highly similar response distributions despite substantial changes in prompt wording. The reversed scale provided a more stringent test of whether LLMs could adapt their responses to fundamentally different question framings. All Reverse Scale responses are reverse-coded back to the standard direction (higher = more satisfied) prior to analysis, ensuring comparability with other questionnaires.

\subsubsection{SWLS: Satisfaction with Life Scale}

The SWLS consists of five items, each rated on a 7-point scale from 1 (strongly disagree) to 7 (strongly agree) \cite{diener1985swls}. The five items are:
\begin{enumerate}
    \item In most ways my life is close to my ideal.
    \item The conditions of my life are excellent.
    \item I am satisfied with my life.
    \item So far I have gotten the important things I want in life.
    \item If I could live my life over, I would change almost nothing.
\end{enumerate}
Responses are summed to produce a total score ranging from 5 to 35. This multi-item format provides potential reliability advantages over single-item measures and tests whether LLMs can maintain consistent persona characteristics across multiple related questions. Total scores are transformed to the 1-10 scale using equipercentile equating (Section~\ref{subsec:equipercentile-equating}).

\subsubsection{OHQ: Oxford Happiness Questionnaire}

The OHQ represents the most comprehensive instrument, comprising 29 items rated on a 6-point scale \cite{hills2002ohq}. The questionnaire assesses multiple facets of happiness and life satisfaction. The 29 items are:
\begin{enumerate}
    \item I don't feel particularly pleased with the way I am. (R)
    \item I am intensely interested in other people.
    \item I feel that life is very rewarding.
    \item I have very warm feelings towards almost everyone.
    \item I rarely wake up feeling rested. (R)
    \item I am not particularly optimistic about the future. (R)
    \item I find most things amusing.
    \item I am always committed and involved.
    \item Life is good.
    \item I do not think that the world is a good place. (R)
    \item I laugh a lot.
    \item I am well satisfied with everything in my life.
    \item I don't think I look attractive. (R)
    \item There is a gap between what I would like to do and what I have done. (R)
    \item I am very happy.
    \item I find beauty in some things.
    \item I always have a cheerful effect on others.
    \item I can fit in (find time for) everything I want to.
    \item I feel that I am not especially in control of my life. (R)
    \item I feel able to take anything on.
    \item I feel fully mentally alert.
    \item I often experience joy and elation.
    \item I don't find it easy to make decisions. (R)
    \item I don't have a particular sense of meaning and purpose in my life. (R)
    \item I feel I have a great deal of energy.
    \item I usually have a good influence on events.
    \item I don't have fun with other people. (R)
    \item I don't feel particularly healthy. (R)
    \item I don't have particularly happy memories of the past. (R)
\end{enumerate}
Items marked with (R) are reverse-scored (items 1, 5, 6, 10, 13, 14, 19, 23, 24, 27, 28, 29). Item responses are averaged to produce a score ranging from 1 to 6, which is then transformed to the 1-10 scale using equipercentile equating (Section~\ref{subsec:equipercentile-equating}). The OHQ's length and complexity test whether LLMs can handle extensive questionnaires requiring sustained attention and nuanced differentiation across many items.

\begin{table}[htbp]
\centering
\caption{Summary of questionnaire characteristics}
\label{tab:questionnaire-summary}
\begin{tabular}{lcccp{5cm}}
\hline
\textbf{Questionnaire} & \textbf{Items} & \textbf{Scale} & \textbf{Equating} & \textbf{Primary Purpose} \\
\hline
Original WVS & 1 & 1-10 direct & None & Standard WVS benchmark \\
Cantril Ladder & 1 & 1-10 direct & None & Metaphorical framing \\
Reverse Scale & 1 & 1-10 reversed & Reverse-code & Cognitive processing test \\
SWLS & 5 & 1-7 each & Equipercentile & Multi-item reliability \\
OHQ & 29 & 1-6 each & Equipercentile & Comprehensive assessment \\
\hline
\end{tabular}
\end{table}

\subsection{Large Language Models}
\label{subsec:llm-models}

Three large language model architectures were employed to generate synthetic survey responses, selected to provide diversity in model size and training approach. This variety enables examination of whether approximation quality varies systematically across different LLM architectures (RQ1).

\textbf{LLaMA 3.1 8B.} This model from Meta AI contains 8 billion parameters and represents a smaller, more efficient architecture \cite{llama3herd2024}.

\textbf{LLaMA 3.3 70B.} Also from Meta AI, this model scales to 70 billion parameters \cite{llama3herd2024}. The comparison between LLaMA 3.1 8B and LLaMA 3.3 70B isolates the effect of model scale within the same architectural family.

\textbf{Qwen 2.5 72B.} Developed by Alibaba Cloud, this 72 billion parameter model provides architectural diversity beyond the LLaMA family \cite{qwen25report2024}. Qwen 2.5 was trained on a corpus with greater representation of Asian languages, which may be relevant for Indonesia.

All three models are multilingual and were accessed via the Academic Cloud inference endpoint, which provides an OpenAI-compatible API. Models were queried with default temperature settings to balance response diversity with consistency. Rate limiting and computational considerations for the generation process are detailed in Section~\ref{subsec:synthetic-data}.

\subsection{Synthetic Data Generation Process}
\label{subsec:synthetic-data}

\subsubsection{Prompting Strategy}

Each synthetic response was generated using an interview-based prompting format where demographic characteristics are introduced through a simulated question-and-answer dialogue. This approach was selected based on recent empirical evidence demonstrating that interview-style prompting reduces stereotyping and improves alignment with human responses compared to alternative prompting formats such as direct role assignment or third-person descriptions \cite{lutz2025prompt}.

The prompt structure follows an interview template that systematically establishes demographic context before presenting the target question:

\begin{quote}
\textit{You are an interviewee. Based on your previous answers, provide an answer to the last question.}

\textit{Interviewer: Where do you live?}\\
\textit{Interviewee: I live in [COUNTRY].}

\textit{Interviewer: What is your income level?}\\
\textit{Interviewee: My income level is [INCOME\_LEVEL].}

\textit{Interviewer: All in all, how would you describe your state of health these days? Would you say it is good, fair, or poor?}\\
\textit{Interviewee: My health is [HEALTH\_LEVEL].}

\textit{Interviewer: [QUESTIONNAIRE TEXT]}\\
\textit{Interviewee:}
\end{quote}

For example, a complete prompt for the United States with medium income and good health using the Original WVS questionnaire would read:

\begin{quote}
\textit{You are an interviewee. Based on your previous answers, provide an answer to the last question.}

\textit{Interviewer: Where do you live?}\\
\textit{Interviewee: I live in the United States.}

\textit{Interviewer: What is your income level?}\\
\textit{Interviewee: My income level is medium.}

\textit{Interviewer: All in all, how would you describe your state of health these days? Would you say it is good, fair, or poor?}\\
\textit{Interviewee: My health is good.}

\textit{Interviewer: All things considered, how satisfied are you with your life as a whole these days? Please respond with a number from 1 to 10, where 10 = completely satisfied and 1 = completely dissatisfied. Reply with a single number from 1 to 10. Do not explain your answer.}\\
\textit{Interviewee:}
\end{quote}

This prompting strategy incorporates several key design decisions justified by both methodological considerations and empirical testing during the thesis proposal preparation phase. First, demographic priming is limited to country, income level, and health status rather than including additional attributes such as gender, age, education, or occupation. This constraint reflects the feature importance analysis results (Section~\ref{subsec:demographic-segments}) and avoids over-specifying personas that might create unrealistic demographic profiles or introduce attributes not systematically measured in the WVS data.

Second, the interview-based format was adopted after systematically comparing multiple prompting approaches during the proposal development stage. Alternative strategies tested included direct role assignment (\textit{``You are a person with...characteristics''}), third-person descriptions (\textit{``Consider a person who...''}), and structured attribute lists (\textit{``Person of income level X, health status Y...''}). The interview format demonstrated superior performance across two critical dimensions: (1) the LLMs produced meaningfully different response distributions for different demographic personas, demonstrating sensitivity to the specified characteristics, and (2) responses exhibited greater diversity within demographic groups while maintaining systematic differences between groups. In contrast, alternative prompting strategies yielded highly similar responses regardless of the demographic characteristics specified, suggesting that the models were not effectively incorporating the persona information into their response generation process.

Third, the interview format aligns with recent findings by Lutz et al. (2025), who demonstrated that interview-style prompts reduce stereotyping, improve semantic diversity, and enhance alignment with human survey responses compared to direct or third-person role adoption formats \cite{lutz2025prompt}. Their systematic evaluation across multiple LLMs and demographic groups showed that interview prompts produce fewer marked words associated with demographic stereotypes and more varied responses within groups, both desirable properties for generating synthetic survey data intended to approximate human response patterns.

Fourth, response format constraints (\textit{``Reply with a single number...Do not explain your answer''}) minimize extraneous text generation and ensure consistent parseable responses. This instruction is particularly important for larger models that may otherwise generate verbose explanations or caveats rather than direct numerical answers.

\subsubsection{Alternative Approaches Not Pursued}

Several alternative prompting strategies were considered but ultimately not implemented. Language-specific prompting, where prompts would be translated into each country's native language (e.g., Spanish for Mexico, Bahasa Indonesia for Indonesia), was rejected for two primary reasons. First, all three LLMs were trained predominantly on English text, with English representing a disproportionately large share of CommonCrawl training data compared to other languages \cite{li2024languageranker}. Recent empirical evidence across educational tasks demonstrates that LLM performance in English consistently exceeds performance in other languages, with average task performance gaps ranging from 5-15 percentage points depending on the language and model \cite{gupta2025multilingual}. Furthermore, using English prompts has been shown to yield equal or superior performance compared to prompts translated into the target language, even when the task content itself is in that language \cite{li2024languageranker}. Second, translation introduces an additional confounding variable that would make it difficult to isolate whether performance differences stem from the LLM's capabilities versus translation quality. The decision to use English prompts throughout therefore prioritizes isolating demographic effects rather than testing multilingual capabilities while leveraging the models' strongest language performance.


\subsubsection{Generation Parameters}

Table~\ref{tab:generation-parameters} summarizes the synthetic data generation structure. The generation process produced responses at multiple levels of aggregation, from individual demographic segments up to the complete dataset.

\begin{table}[htbp]
\centering
\caption{Synthetic data generation structure}
\label{tab:generation-parameters}
\begin{tabular}{lrl}
\hline
\textbf{Level} & \textbf{Count} & \textbf{Calculation} \\
\hline
Responses per segment & 30 & Fixed sample size \\
Segments per country & 9 & 3 income $\times$ 3 health \\
Responses per country & 270 & 30 $\times$ 9 segments \\
Countries & 4 & USA, Indonesia, Netherlands, Mexico \\
Responses per model-questionnaire & 1,080 & 270 $\times$ 4 countries \\
Models & 3 & LLaMA 3.1 8B, LLaMA 3.3 70B, Qwen 2.5 72B \\
Questionnaires & 5 & Original WVS, Cantril, Reverse, SWLS, OHQ \\
\textbf{Total synthetic responses} & \textbf{16,200} & 1,080 $\times$ 3 models $\times$ 5 questionnaires \\
\hline
\end{tabular}
\end{table}

All responses were validated to ensure numeric values within the valid range for each questionnaire. Invalid or missing responses triggered automatic regeneration, though such cases were rare.

\subsubsection{Weighting Procedure}

To enable valid distributional comparisons, each synthetic response was assigned a matching weight (\texttt{weight\_joint}) that aligns the synthetic demographic composition with the real WVS distribution. Synthetic data generation employed uniform sampling (30 responses per segment), whereas the real WVS exhibits unequal segment sizes reflecting natural population heterogeneity. Without weighting, distributional comparisons would confound true approximation quality with demographic composition differences.

The weighting procedure involves two steps. First, segment-level adjustment weights correct for differences in group sizes between synthetic and real data: for each demographic segment, the adjustment weight equals the ratio of the segment's representation in real WVS data to its representation in synthetic data, scaled to the overall sample size. Second, these adjustment weights are multiplied by the original WVS survey weights (\texttt{W\_WEIGHT}) to incorporate the complex survey design. The final synthetic weight (\texttt{weight\_joint}) thus reflects both demographic composition alignment and the WVS sampling weights. This two-step procedure ensures that when synthetic distributions are aggregated (e.g., at the country level), the demographic composition mirrors the weighted real WVS distribution (with survey weights applied), enabling unbiased distributional distance calculations.

\subsection{Equipercentile Equating}
\label{subsec:equipercentile-equating}

Equipercentile equating was employed to transform SWLS (1-7 scale) and OHQ (1-6 scale) scores to a common 1-10 scale, enabling direct distributional comparisons across all five questionnaires \cite{kolen2014equating, gesis2023equating}. Original WVS, Cantril Ladder, and Reverse Scale variants use 1-10 scales directly. Simple linear rescaling would impose arbitrary distributional shapes; equipercentile equating instead preserves percentile ranks, empirically determining score correspondence that maintains distributional characteristics.

\subsubsection{Procedure}

The core intuition behind equipercentile equating is straightforward: if a respondent scores at the 75th percentile on the SWLS scale, they should be mapped to the score corresponding to the 75th percentile on the WVS 1-10 scale. This preserves relative position in the distribution rather than imposing arbitrary linear transformations.

To limit data leakage, a random 10\% subset of real WVS data (sampled with seed 42) was used exclusively to construct the mapping, while the full 100\% was used for subsequent comparisons. The procedure follows three steps:

\textbf{Step 1: Construct the target percentile curve.} Using the 10\% WVS training subset ($D_{train}$), build the weighted Cumulative Distribution Function (CDF):
\begin{equation}
F_{target}(y) = \sum_{i \in D_{train}: r_i \leq y} v_i \Big/ \sum_{j \in D_{train}} v_j
\end{equation}
where $r_i$ are life satisfaction scores (1-10) and $v_i$ are sampling weights. Linear interpolation between observed scores produces a smooth percentile curve, solving the ``continuization'' challenge of discrete response categories.

\textbf{Step 2: Map source scores to percentiles, then to target scores.} For SWLS (1-7) and OHQ (1-6), assume uniform distributions over source categories. Each integer $x$ corresponds to a percentile under this uniform assumption. This percentile is then mapped to the target 1-10 scale by inverting $F_{target}$:
\begin{equation}
y_{equated}(x) = F_{target}^{-1}\left(\text{Percentile}_{uniform}(x)\right)
\end{equation}
Results are rounded to integers and clipped to [1,10], producing fixed lookup tables with 7 SWLS mappings and 6 OHQ mappings.

\textbf{Step 3: Apply fixed mappings to all synthetic data.} The lookup tables are applied uniformly to all synthetic responses, generating \texttt{life\_satisfaction\_equated} (SWLS) and \texttt{ohq\_equated} (OHQ). All five questionnaires now share a common 1-10 scale for distributional comparisons.

\subsubsection{Implementation and Validation}

The procedure was implemented using SciPy's \texttt{interp1d} interpolation function with weighted percentile calculations. The 10\%--90\% split balances competing considerations: using only 10\% for mapping construction limits data leakage that would artificially inflate equated scale performance, while using 100\% for evaluation maximizes statistical power. This is methodologically sound because the mapping is a fixed preprocessing step, not adapted during evaluation.

Validation checks confirmed: (1) no endpoint bunching at scores 1 or 10, (2) monotonicity preserved (higher source scores map to higher target scores), and (3) no systematic distributional distortions relative to real WVS data.

\subsection{Demographic Segmentation}
\label{subsec:demographic-segments}

The subgroup analysis (Section~\ref{sec:subgroup-results}) examines approximation quality across 36 demographic segments created through cross-classification of country, income level, and health status. This segmentation structure was designed to capture theoretically important demographic variation while maintaining sufficient sample sizes for reliable distributional comparisons.

\subsubsection{Cross-Classification Structure}

Demographic segments are defined by the combination of three factors:
\begin{itemize}
    \item \textbf{Countries}: 4 levels (USA, Indonesia, Netherlands, Mexico)
    \item \textbf{Income levels}: 3 levels (Low, Medium, High)
    \item \textbf{Health status}: 3 levels (Poor, Fair, Good)
\end{itemize}

The full factorial design produces $4 \times 3 \times 3 = 36$ unique demographic segments. Each segment represents a distinct subpopulation defined by specific combinations of geographic location, socioeconomic position, and health condition. For example, one segment comprises individuals in the United States with high income and good health, while another represents individuals in Indonesia with low income and poor health. The recoding procedures for income and health status are described in Section~\ref{subsec:wvs-data}.

As described in Section~\ref{subsec:wvs-data}, income and health were selected based on Random Forest feature importance analysis (see Figure~\ref{fig:feature-importance}). Country captures cross-national variation, while income and health represent the most consequential within-country predictors. This three-factor design balances substantive coverage with statistical power.

\subsubsection{Sample Structure Summary}

The 36-segment structure defines the analytical scope for the subgroup analysis (Section~\ref{sec:subgroup-results}). Within each segment, distributional comparisons are conducted across all model-questionnaire combinations (3 models $\times$ 5 questionnaires = 15 configurations per segment). This produces 540 total segment-level distributional comparisons ($36 \times 15 = 540$).

For country-level analysis (Section~\ref{sec:country-level-results}), segments are aggregated within each country to produce nationally representative distributions. This aggregation uses survey weights (\texttt{W\_WEIGHT} for real data, \texttt{weight\_joint} for synthetic data) to account for differential segment sizes and ensure valid population-level estimates. Country-level comparisons total 60 ($4$ countries $\times$ $3$ models $\times$ $5$ questionnaires = $60$).

Table~\ref{tab:sample-structure} summarizes the sample structure across demographic segments and analytical levels.

\begin{table}[htbp]
\centering
\caption{Sample structure for subgroup and country-level analyses}
\label{tab:sample-structure}
\begin{tabular}{lcc}
\hline
\textbf{Analytical Unit} & \textbf{Number of Units} & \textbf{Comparisons per Unit} \\
\hline
Demographic segment & 36 & 15 (3 models × 5 questionnaires) \\
Country & 4 & 15 (3 models × 5 questionnaires) \\
\hline
\textbf{Total comparisons} & \multicolumn{2}{c}{600 (540 segment + 60 country)} \\
\hline
\end{tabular}
\end{table}

%-----------------------------------------------------------
\section{Distributional Distance Metrics}
\label{sec:distance-metrics}

Two complementary metrics quantify distributional differences between synthetic and real life satisfaction distributions: Wasserstein distance (primary metric) and Kolmogorov-Smirnov statistic (secondary validation).

\subsection{Wasserstein Distance}
\label{subsec:wasserstein}

The Wasserstein distance, also known as earth mover's distance, quantifies the minimum ``work'' required to transform one distribution into another \cite{villani2009optimal}. Formally, the first-order Wasserstein distance between distributions $P$ and $Q$ with CDFs $F_P$ and $F_Q$ is:

\begin{equation}
W_1(P, Q) = \int_{-\infty}^{\infty} |F_P(x) - F_Q(x)| \, dx
\end{equation}

For discrete weighted observations, this reduces to computing the area between empirical CDFs.

\textbf{Advantages over alternatives.} The Wasserstein distance was selected over Kullback-Leibler divergence, Jensen-Shannon divergence, and total variation distance for five reasons: (1) true metric properties (symmetry, triangle inequality), (2) interpretability in original scale units (life satisfaction points), (3) sensitivity to full distributional shape beyond central tendency, (4) robustness to zero probabilities, and (5) natural accommodation of survey weights.

\textbf{Implementation.} All calculations used SciPy's Wasserstein distance function \cite{virtanen2020scipy}, applying WVS sampling weights to real data and matching weights to synthetic data. The function returns scalar $W$ values representing average distance in life satisfaction units. Table~\ref{tab:wasserstein-interpretation} provides interpretation guidelines.

\begin{table}[htbp]
\centering
\caption{Interpretation guidelines for Wasserstein distance values}
\label{tab:wasserstein-interpretation}
\begin{tabular}{lll}
\hline
\textbf{Wasserstein ($W$)} & \textbf{Assessment} & \textbf{Interpretation} \\
\hline
0.0--0.5 & Excellent & Near-perfect approximation \\
0.5--1.0 & Good & Minor differences \\
1.0--1.5 & Moderate & Noticeable differences \\
1.5--2.0 & Fair & Substantial differences \\
$>$ 2.0 & Poor & Large discrepancies \\
\hline
\end{tabular}
\end{table}

\subsection{Kolmogorov-Smirnov Statistic}
\label{subsec:ks-statistic}

The Kolmogorov-Smirnov (KS) statistic provides secondary validation \cite{massey1951ks}. While Wasserstein distance integrates total area between CDFs, KS identifies maximum point-wise divergence:

\begin{equation}
D_{KS} = \sup_{x} |F_P(x) - F_Q(x)|
\end{equation}

where $\sup$ denotes the supremum (maximum), and $F_P$, $F_Q$ are CDFs. Values range from 0 (identical) to 1 (non-overlapping).

\textbf{Dual-metric validation.} Computing both metrics serves three purposes: (1) \textit{convergent validity}---correlation $r = 0.755$ ($p < 0.001$) across 540 comparisons confirms both metrics identify good vs. poor approximations; (2) \textit{complementary information}---Wasserstein quantifies overall shift, KS locates maximum divergence point; (3) \textit{robustness check}---58.3\% of segments show identical best configurations under both metrics, with disagreements indicating multiple viable strategies.

\textbf{Decision rule.} Given superior interpretability (life satisfaction units vs. abstract probabilities), primary analyses use Wasserstein distance. KS statistics are reported in tables and discussed when disagreements provide informative insights about multiple good approximations.

%-----------------------------------------------------------
\section{Variance Decomposition Analysis}
\label{sec:variance-decomposition}

The variance decomposition analysis addresses the fundamental question: ``Which factors matter most for distributional approximation quality?'' \cite{searle1992variance}. Across the 540 segment-level comparisons (36 demographic segments $\times$ 5 questionnaires $\times$ 3 models), Wasserstein distances vary considerably. The variance decomposition quantifies the relative importance of five factors in explaining this variation: country, income level, health status, questionnaire type, and LLM model.

\subsection{Rationale and Purpose}
\label{subsec:variance-rationale}

The variance decomposition serves to quantify how much each factor spreads out the distribution of Wasserstein distances. A factor with high relative importance is one where knowing the factor level substantially reduces uncertainty about the expected approximation quality. For example, if certain countries consistently yield better approximations than others, the ``country'' factor will explain a large proportion of variance; conversely, if approximation quality is similar across countries, the country factor will contribute little to total variance.

This approach is particularly suitable for the study design, which involves an unbalanced factorial structure with unequal numbers of factor levels (3 models, 4 countries, 5 questionnaires, 3 income levels, 3 health levels). Traditional ANOVA-based eta-squared estimates require balanced designs and assume orthogonal factors without interactions. The variance decomposition method employed here makes no such assumptions, instead providing a descriptive measure of how much each factor spreads out group means relative to total variation.

\subsection{Method: Between-Group Variance Ratio}
\label{subsec:variance-method}

For each factor $F$ (e.g., health status), the relative importance is computed through the following four-step procedure:

\textbf{Step 1: Calculate total variance.} Compute the variance of Wasserstein distances across all 540 segment-level comparisons:

\begin{equation}
\sigma^2_{total} = \text{Var}(W)
\end{equation}

where $W$ represents the vector of all Wasserstein distance values.

\textbf{Step 2: Calculate group means.} For each level $k$ of factor $F$, compute the mean Wasserstein distance:

\begin{equation}
\mu_k = \mathbb{E}[W \mid F = k]
\end{equation}

For example, if $F$ represents health status with three levels (Good, Fair, Poor), this step computes three group means: $\mu_{Good}$, $\mu_{Fair}$, and $\mu_{Poor}$.

\textbf{Step 3: Calculate between-group variance.} Quantify how much the group means spread out around the grand mean:

\begin{equation}
\sigma^2_{between} = \frac{1}{K} \sum_{k=1}^{K} (\mu_k - \mu_{grand})^2
\end{equation}

where $K$ is the number of factor levels and $\mu_{grand}$ is the overall mean across all observations.

\textbf{Step 4: Calculate relative importance.} Express the between-group variance as a percentage of total variance:

\begin{equation}
RI_F = \left( \frac{\sigma^2_{between}}{\sigma^2_{total}} \right) \times 100\%
\end{equation}

Higher values of $RI_F$ indicate that the factor explains more of the observed variation in Wasserstein distances. A value of $RI_F = 30\%$ means that 30\% of the total variance in approximation quality is attributable to differences between the factor's levels.

\subsubsection{Mathematical Justification}
\label{subsubsec:variance-justification}

The relative importance measure quantifies the extent to which knowing a factor's level reduces uncertainty about the expected Wasserstein distance. This interpretation is grounded in the variance decomposition principle:

\begin{equation}
\text{Total Variance} = \text{Between-Group Variance} + \text{Within-Group Variance}
\end{equation}

The ratio $\sigma^2_{between} / \sigma^2_{total}$ thus captures the proportion of total variability that is attributable to group membership. Factors with large between-group variance (i.e., widely separated group means) explain more of the total variation than factors with small between-group variance (i.e., similar group means).

Note that this measure is a descriptive index of relative importance, not a formal effect size estimate such as eta-squared ($\eta^2$) from ANOVA. The relative importance percentages do not sum to 100\% because factors are non-orthogonal and the measure captures average effects across all factor combinations rather than unique variance explained. This approach suits exploratory analysis where the goal is identifying which factors have the largest average effects, providing intuitive interpretation without requiring strong assumptions of traditional ANOVA models.

To assess whether the observed differences in approximation quality across factor levels are statistically significant, the F-test from one-way Analysis of Variance (ANOVA) is employed for each factor independently \parencite{fisher1925statistical}. The F-statistic quantifies the ratio of between-group variance to within-group variance; a large F-value indicates that group means differ more than would be expected by chance. Factors with p $<$ .05 are considered statistically significant.

The variance decomposition is applied to the 540 segment-level comparisons (36 demographic segments $\times$ 5 questionnaires $\times$ 3 models). For each segment, the best configuration is identified by determining which model-questionnaire combination yields minimum Wasserstein distance and checking whether the same configuration also minimizes KS statistic, with agreement providing convergent evidence.

%-----------------------------------------------------------
\section{Score-Level Analysis}
\label{sec:score-level-method}

The preceding analyses examine approximation quality through aggregate distributional metrics that compare entire distributions. This section introduces a complementary approach that examines approximation quality at the level of individual satisfaction scores, addressing which specific life satisfaction scores (1--10) are predicted more accurately by LLMs. Aggregate metrics such as Wasserstein distance may obscure systematic patterns in how LLMs approximate specific response categories; for instance, if LLMs consistently over-predict moderate satisfaction levels while under-predicting extreme levels, this pattern would be partially captured by aggregate metrics but not explicitly identified.

For each life satisfaction score $s \in \{1, 2, \ldots, 10\}$, let $p_s^{real}$ denote the weighted proportion of real WVS respondents selecting score $s$, and $p_s^{synth}$ denote the corresponding proportion in synthetic data. The primary metric is the absolute error:

\begin{equation}
\text{AbsError}_s = |p_s^{real} - p_s^{synth}|
\label{eq:score-abs-error}
\end{equation}

This metric quantifies the discrepancy between real and synthetic proportions at each score level, expressed in probability units (range: 0 to 1). To assess prediction bias direction, the signed difference $\text{Difference}_s = p_s^{synth} - p_s^{real}$ is also computed, where positive values indicate over-prediction and negative values indicate under-prediction.

Score-level metrics are computed for each combination of model, questionnaire, and country, yielding 60 distributional comparisons and 600 total score-level observations (10 scores per comparison). Proportions incorporate survey weights for real WVS data and matching weights for synthetic data. Results are aggregated at overall, questionnaire-level, and model-level to distinguish universal patterns from conditional patterns that depend on specific questionnaires or models.

%-----------------------------------------------------------
\section{Ensemble Approaches}
\label{sec:ensemble-approaches}

The ensemble analysis addresses RQ3 by testing whether combining results from multiple questionnaires improves alignment with real survey responses beyond what any single questionnaire achieves. Two ensemble strategies are evaluated, both grounded in the principle that different measurement instruments may capture complementary aspects of life satisfaction, and their combination may reduce measurement-specific biases.

\subsection{Rationale for Ensemble Methods}
\label{subsec:ensemble-rationale}

Ensemble methods are widely used in machine learning and psychometrics to improve prediction accuracy and reduce error variance \cite{dietterich2000ensemble}. In the context of synthetic survey data, different questionnaire formats may introduce systematic biases---for example, metaphorical framing (Cantril Ladder) may elicit different response patterns than direct assessment (Original WVS), and multi-item scales (SWLS, OHQ) may capture different facets of the underlying construct than single-item measures.

By combining predictions across multiple questionnaires, ensemble approaches can potentially reduce questionnaire-specific measurement error through averaging, capture complementary information from different measurement approaches, and improve robustness by not relying on any single instrument's performance. The ensemble analysis excludes the Reverse Scale questionnaire due to its consistently poor performance identified in preliminary analyses; including poorly-performing instruments would degrade ensemble quality rather than improve it.

\subsection{KS-Based Optimal Weighting}
\label{subsec:ks-weighting}

The KS-based ensemble assigns differential weights to questionnaires based on their individual performance, giving higher weight to better-performing instruments. This approach assumes that questionnaires demonstrating stronger alignment with real data should contribute more to the combined prediction. Importantly, this method requires prior knowledge of questionnaire performance from existing human survey data; in real-world applications where no human responses are available, such performance-based weighting would not be feasible. The KS statistic was chosen over Wasserstein distance for weighting because it is naturally bounded between 0 and 1, enabling direct conversion to weights without requiring arbitrary normalization of unbounded distance metrics.

\subsubsection{Weight Calculation}

For each model $m$, questionnaire weights are calculated using inverse KS statistics averaged across countries:

\begin{equation}
w_q^{(m)} = \frac{1 - \overline{KS}_q^{(m)}}{\sum_{q'=1}^{Q} (1 - \overline{KS}_{q'}^{(m)})}
\label{eq:ks-weights}
\end{equation}

where $\overline{KS}_q^{(m)}$ is the mean KS statistic for questionnaire $q$ using model $m$ across all four countries, and $Q = 4$ is the number of questionnaires in the ensemble (Original WVS, Cantril Ladder, SWLS, OHQ).

The weight formula has intuitive interpretation: questionnaires with lower KS statistics (better alignment) receive higher weights. The denominator normalizes weights to sum to 1, ensuring the ensemble produces valid probability distributions.

\subsubsection{Ensemble Distribution Construction}

For each model-country combination, the KS-weighted ensemble distribution is constructed by:

\textbf{Step 1: Obtain individual distributions.} For each of the four questionnaires, compute the weighted histogram of synthetic life satisfaction responses on the 1-10 scale, using the matching weights (\texttt{weight\_joint}) to ensure demographic representativeness.

\textbf{Step 2: Apply questionnaire weights.} Combine the four questionnaire distributions using the KS-based weights:

\begin{equation}
P_{ensemble}(x) = \sum_{q=1}^{Q} w_q \cdot P_q(x)
\label{eq:ensemble-distribution}
\end{equation}

where $P_q(x)$ is the probability mass at satisfaction level $x$ for questionnaire $q$, and $w_q$ are the KS-based weights from Equation~\ref{eq:ks-weights}.

\textbf{Step 3: Compute distributional distance.} Calculate Wasserstein distance and KS statistic between the ensemble distribution and the real WVS distribution for the corresponding country.

\subsection{Equal-Weight Averaging}
\label{subsec:equal-weight}

The equal-weight ensemble treats all four questionnaires as equally valid measurement approaches, assigning uniform weights:

\begin{equation}
w_q = \frac{1}{Q} = 0.25 \quad \text{for all } q \in \{1, 2, 3, 4\}
\label{eq:equal-weights}
\end{equation}

This approach makes no assumptions about differential questionnaire quality and provides a simple baseline for comparison with the performance-based KS weighting. Equal weighting is robust to potential overfitting that could occur if KS-based weights are optimized on the same data used for evaluation.

The ensemble distribution construction follows the same procedure as KS-based weighting (Section~\ref{subsec:ks-weighting}), substituting equal weights for performance-based weights in Equation~\ref{eq:ensemble-distribution}.

%-----------------------------------------------------------
\section{Software and Implementation}
\label{sec:software}

All data processing, statistical analysis, and visualization were performed using Python 3.11.3. The computational environment relied on the following key packages:

\begin{itemize}
    \item \textbf{Data manipulation}: \texttt{pandas} (2.1.0), \texttt{numpy} (1.25.2)
    \item \textbf{Statistical analysis}: \texttt{scipy} (1.11.3) for Wasserstein distance and KS statistic computation, \texttt{scikit-learn} (1.3.0) for feature importance analysis
    \item \textbf{Visualization}: \texttt{matplotlib} (3.7.2), \texttt{seaborn} (0.12.2)
    \item \textbf{LLM inference}: \texttt{openai} (1.3.0) for API-based synthetic response generation
    \item \textbf{Additional utilities}: \texttt{tqdm} (4.66.1) for progress tracking during generation, \texttt{umap-learn} (0.5.4) for dimensionality reduction visualizations
\end{itemize}

\textbf{Wasserstein distance computation.} All distributional comparisons used SciPy's Wasserstein distance function, which implements the first-order Wasserstein metric with support for weighted observations.

\textbf{Code availability.} Complete analysis code, including data preprocessing, synthetic data generation, distributional comparisons, variance decomposition, and visualization scripts, along with all necessary data, is available at \url{https://github.com/milanataova/master-thesis}.

%-----------------------------------------------------------
% End of Chapter 4

% Chapter 5: Results
% This chapter presents findings from the three analytical approaches: country-level evaluation, subgroup analysis, and ensemble methods

\chapter{Results}
\label{ch:results}

\section{Overview of Results}
\label{sec:results-overview}

This chapter presents the empirical findings from the three analytical approaches described in Chapter~\ref{ch:method}. The analysis addresses the three research questions through progressively detailed examination of distributional alignment between synthetic and real survey responses. Section~\ref{sec:country-level-results} presents country-level results aggregated across demographic subgroups, providing a broad assessment of approximation quality across the 60 model-questionnaire-country combinations. Section~\ref{sec:subgroup-results} examines the 540 segment-level comparisons, employing variance decomposition to identify which factors most strongly influence approximation quality. Section~\ref{sec:ensemble-results} evaluates whether ensemble methods combining multiple questionnaires can improve alignment beyond individual scales.

Throughout this chapter, approximation quality is quantified using Wasserstein distance as the primary metric, with lower values indicating better alignment between synthetic and real distributions. Kolmogorov-Smirnov (KS) statistics are reported as secondary validation. All comparisons incorporate survey weights to ensure nationally representative estimates.

%-----------------------------------------------------------
\section{Country-Level Evaluation}
\label{sec:country-level-results}

The country-level evaluation addresses RQ1 (overall approximation quality) and RQ2 (questionnaire differences) through 60 distributional comparisons aggregated at the national level. Across all comparisons, mean Wasserstein distance was 1.44 (SD = 0.72), indicating that on average, synthetic distributions differed from real WVS distributions by approximately 1.4 points on the 10-point life satisfaction scale. However, approximation quality varied substantially, ranging from excellent alignment (W = 0.16, Netherlands with LLaMA 3.1 8B using Original WVS) to poor alignment (W = 3.28, Netherlands with LLaMA 3.3 70B using REVERSE). These results suggest that LLMs can achieve reasonable approximations under favorable conditions, but performance is highly dependent on the specific combination of model, questionnaire, and target population.

Contrary to expectations that larger models would yield better approximations, the smallest model (LLaMA 3.1 8B) substantially outperformed its larger counterparts (Table~\ref{tab:model-country-results}). LLaMA 3.1 8B achieved the lowest mean Wasserstein distance (W = 1.23) and KS statistic (KS = 0.39), followed by LLaMA 3.3 70B (W = 1.46, KS = 0.40) and Qwen 2.5 72B (W = 1.63, KS = 0.46). This pattern persisted across all four countries and most questionnaire types, suggesting a systematic advantage rather than an artifact of specific conditions.

\begin{table}[htbp]
\centering
\caption{Country-level performance by model (averaged across questionnaires and countries)}
\label{tab:model-country-results}
\begin{tabular}{lccc}
\hline
\textbf{Model} & \textbf{Parameters} & \textbf{Mean W (SD)} & \textbf{Mean KS (SD)} \\
\hline
LLaMA 3.1 8B & 8B & 1.23 (0.48) & 0.39 (0.17) \\
LLaMA 3.3 70B & 70B & 1.46 (0.88) & 0.40 (0.12) \\
Qwen 2.5 72B & 72B & 1.63 (0.72) & 0.46 (0.11) \\
\hline
\end{tabular}
\end{table}

Approximation quality also varied substantially across questionnaire types (Table~\ref{tab:questionnaire-country-results}), directly addressing RQ2. The Original WVS questionnaire achieved the best average Wasserstein distance (W = 0.95), followed closely by SWLS (W = 1.03), both within the ``good'' approximation range. CANTRIL and OHQ showed moderate performance (W = 1.24 and W = 1.46 respectively), suggesting that metaphorical framing and multi-item complexity introduce additional approximation challenges. The REVERSE questionnaire performed markedly worse (W = 2.53), with Wasserstein distances 166\% higher than Original WVS, indicating that LLMs struggle to correctly process reversed scale orientations.

\begin{table}[htbp]
\centering
\caption{Country-level performance by questionnaire type (averaged across models and countries)}
\label{tab:questionnaire-country-results}
\begin{tabular}{lccl}
\hline
\textbf{Questionnaire} & \textbf{Mean W (SD)} & \textbf{Mean KS (SD)} & \textbf{Assessment} \\
\hline
Original WVS & 0.95 (0.40) & 0.36 (0.16) & Good \\
SWLS & 1.03 (0.18) & 0.34 (0.08) & Good \\
Cantril & 1.24 (0.57) & 0.39 (0.14) & Moderate \\
OHQ & 1.46 (0.19) & 0.45 (0.12) & Moderate \\
Reverse & 2.53 (0.67) & 0.53 (0.06) & Poor \\
\hline
\end{tabular}
\end{table}

Geographic variation was also evident: the Netherlands showed the best overall approximation (W = 1.21), while Mexico showed the highest approximation difficulty (W = 1.60), representing a 32\% performance gap (Table~\ref{tab:geographic-results}). This pattern likely reflects differential representation in LLM training corpora, where Western countries may have more extensive representation.

\begin{table}[htbp]
\centering
\caption{Country-level performance by country (averaged across models and questionnaires)}
\label{tab:geographic-results}
\begin{tabular}{lccc}
\hline
\textbf{Country} & \textbf{Mean W} & \textbf{Mean KS} & \textbf{Rank} \\
\hline
Netherlands & 1.21 & 0.38 & 1 (Best) \\
USA & 1.41 & 0.40 & 2 \\
Indonesia & 1.54 & 0.40 & 3 \\
Mexico & 1.60 & 0.47 & 4 (Hardest) \\
\hline
\end{tabular}
\end{table}

\subsection{Distributional Comparisons}
\label{subsec:distributional-comparisons}

This section examines the actual life satisfaction distributions, comparing real WVS data with synthetic responses generated by each LLM-questionnaire combination. Figures~\ref{fig:distribution-usa}--\ref{fig:distribution-mex} display these distributions, with the black line representing real WVS data and colored lines showing synthetic distributions from each questionnaire type. Each panel corresponds to one LLM. These visualizations illustrate what the Wasserstein distance metric captures: the amount of probability mass that must be moved to transform one distribution into the other.

\subsubsection*{United States}

\begin{figure}[htbp]
    \centering
    \includegraphics[width=\textwidth]{figures/fig_distribution_USA.png}
    \caption{Life satisfaction distributions for the United States. Black line shows real WVS data; colored lines show synthetic distributions from each questionnaire type. Each panel represents one LLM. Original WVS and SWLS track the real distribution most closely, while REVERSE shows systematic deviation; LLaMA 3.1 8B achieves the tightest overall alignment.}
    \label{fig:distribution-usa}
\end{figure}

The real US distribution (Figure~\ref{fig:distribution-usa}) exhibits a unimodal shape with a peak around satisfaction level 8. Original WVS and SWLS track this distribution most closely, while REVERSE shows systematic deviation, particularly for LLaMA 3.3 70B, where synthetic responses cluster at lower satisfaction levels. LLaMA 3.1 8B achieves the tightest alignment overall.

\subsubsection*{Netherlands}

\begin{figure}[htbp]
    \centering
    \includegraphics[width=\textwidth]{figures/fig_distribution_NLD.png}
    \caption{Life satisfaction distributions for the Netherlands. Black line shows real WVS data; colored lines show synthetic distributions from each questionnaire type. Each panel represents one LLM. The Netherlands shows the best overall alignment, with LLaMA 3.1 8B achieving near-perfect approximation using Original WVS (W = 0.16); REVERSE again produces the largest deviations.}
    \label{fig:distribution-nld}
\end{figure}

The Netherlands (Figure~\ref{fig:distribution-nld}) shows high concentration at satisfaction levels 7--8 with sharp drop-offs at both extremes. This represents the best-approximated country, with LLaMA 3.1 8B achieving near-perfect alignment using Original WVS (W = 0.16). The relatively homogeneous Dutch satisfaction distribution appears easier for LLMs to model, as even larger models achieve reasonable approximation. REVERSE again produces the largest deviations.

\subsubsection*{Indonesia}

\begin{figure}[htbp]
    \centering
    \includegraphics[width=\textwidth]{figures/fig_distribution_IDN.png}
    \caption{Life satisfaction distributions for Indonesia. Black line shows real WVS data; colored lines show synthetic distributions from each questionnaire type. Each panel represents one LLM. LLaMA 3.1 8B with Original WVS best reproduces the peak location, while REVERSE shows nearly inverted patterns with peaks at levels 3--4 instead of 7--8.}
    \label{fig:distribution-idn}
\end{figure}

Indonesia (Figure~\ref{fig:distribution-idn}) displays a pronounced right skew with a dominant peak at satisfaction level 8. Synthetic distributions generally capture this rightward tendency, with LLaMA 3.1 8B and Original WVS most accurately reproducing the peak location. OHQ produces more dispersed distributions, while REVERSE shows nearly inverted patterns, with some combinations showing peaks at levels 3--4 instead of 7--8.

\subsubsection*{Mexico}

\begin{figure}[htbp]
    \centering
    \includegraphics[width=\textwidth]{figures/fig_distribution_MEX.png}
    \caption{Life satisfaction distributions for Mexico. Black line shows real WVS data; colored lines show synthetic distributions from each questionnaire type. Each panel represents one LLM. Mexico presents the highest approximation difficulty, with synthetic distributions consistently underestimating concentration at the highest satisfaction levels (9--10).}
    \label{fig:distribution-mex}
\end{figure}

Mexico (Figure~\ref{fig:distribution-mex}) presents the most challenging case. The real distribution shows strong right skew with concentration at levels 8--10, reflecting characteristically high self-reported satisfaction. Synthetic distributions consistently underestimate concentration at the highest levels (9--10), suggesting LLMs struggle to model populations with extreme response tendencies. REVERSE performs particularly poorly, with Qwen 2.5 72B producing an almost uniform distribution.

\subsubsection*{Summary of Distributional Patterns}

Figure~\ref{fig:wasserstein-by-country} summarizes the quantitative differences observed in the distributional comparisons, presenting Wasserstein distances for all model-questionnaire combinations. The bar chart confirms the patterns observed in the distribution plots: REVERSE consistently shows the highest distances across all conditions, Original WVS and SWLS typically achieve the lowest distances, and LLaMA 3.1 8B (leftmost group in each panel) generally outperforms larger models. Corresponding KS statistics are presented in Appendix~\ref{app:ks-results}.

\begin{figure}[htbp]
    \centering
    \includegraphics[width=\textwidth]{figures/fig_wasserstein_by_country.png}
    \caption{Wasserstein distance by model and questionnaire across countries. Each panel represents one country, with models grouped on the x-axis and questionnaire types distinguished by color. Lower bars indicate better approximation. The REVERSE questionnaire (orange) consistently shows the highest Wasserstein distances, while Original WVS (green) and SWLS (purple) typically achieve the best performance.}
    \label{fig:wasserstein-by-country}
\end{figure}

To further examine the structure of distributional similarity, UMAP (Uniform Manifold Approximation and Projection) dimensionality reduction was applied to the life satisfaction distributions. Each data point in Figure~\ref{fig:umap-combined} represents a country-questionnaire combination, with the input being a 10-dimensional vector of weighted relative frequencies for satisfaction scores 1--10. Points that are close together in the UMAP space have similar distributional shapes. Colors indicate questionnaire type (black = real WVS, green = Original WVS, orange = REVERSE, blue = CANTRIL, purple = SWLS, red = OHQ), while marker shapes indicate country (circle = USA, square = Indonesia, triangle = Netherlands, diamond = Mexico).

\begin{figure}[htbp]
    \centering
    \begin{minipage}{0.48\textwidth}
        \centering
        \includegraphics[width=\textwidth]{figures/fig_umap_llama31_8b.png}
        \centerline{(a) LLaMA 3.1 8B}
    \end{minipage}
    \hfill
    \begin{minipage}{0.48\textwidth}
        \centering
        \includegraphics[width=\textwidth]{figures/fig_umap_llama33_70b.png}
        \centerline{(b) LLaMA 3.3 70B}
    \end{minipage}

    \vspace{0.5cm}

    \begin{minipage}{0.48\textwidth}
        \centering
        \includegraphics[width=\textwidth]{figures/fig_umap_qwen25_72b.png}
        \centerline{(c) Qwen 2.5 72B}
    \end{minipage}
    \caption{UMAP projections of aggregated life satisfaction distributions by country and LLM. Points cluster primarily by questionnaire type (color) rather than country (shape), indicating that measurement instrument choice has a stronger influence on distributional similarity than geographic context.}
    \label{fig:umap-combined}
\end{figure}

The projections show that distributions cluster primarily by questionnaire type rather than by country. Points of the same color tend to group together regardless of their shape, while points of the same shape are dispersed across different regions of the embedding space. This strong clustering by questionnaire type may seem to contradict the variance decomposition findings in Section~\ref{sec:subgroup-results}, where questionnaire type explains only 6.9\% of variance in approximation quality---less than health status (11.0\%) or model choice (8.0\%). However, these analyses measure different phenomena: the UMAP clustering indicates that different questionnaires produce distinctly different response distributions, while variance decomposition measures how well those distributions align with real data. In other words, all questionnaires produce different results from each other, but this distinctness does not necessarily translate to better or worse approximation of human responses.

The REVERSE questionnaire (orange) forms a distinct cluster separated from other questionnaires across all three models, consistent with its systematically poorer distributional alignment observed in the Wasserstein analysis. The real WVS data points (black) cluster near the Original WVS and SWLS synthetic distributions, suggesting that these questionnaires produce response patterns most similar to actual human responses. The four country markers for each questionnaire type remain relatively close together, indicating that within-questionnaire variation across countries is smaller than between-questionnaire variation within countries.

%-----------------------------------------------------------
\section{Subgroup Analysis}
\label{sec:subgroup-results}

The subgroup analysis extends the country-level findings by examining approximation quality across 36 demographic segments (4 countries $\times$ 3 income levels $\times$ 3 health statuses), yielding 540 unique comparisons (36 segments $\times$ 5 questionnaires $\times$ 3 models). This granular analysis identifies which demographic factors most strongly influence approximation quality and reveals systematic patterns that aggregate analyses may obscure.

Figure~\ref{fig:specification-curve} provides an overview of all 540 segment-level comparisons, displaying Wasserstein distances sorted from highest (worst approximation) to lowest (best approximation). The specification curve reveals substantial variation in approximation quality across configurations, with Wasserstein distances ranging from 0.35 to 7.60 (mean = 2.09). The bottom panel indicates which factor levels correspond to each specification, enabling visual identification of systematic patterns. Notably, Reverse Scale questionnaire configurations (orange) cluster predominantly among the worst-performing specifications, while Netherlands and LLaMA 3.1 8B dominate the best-performing region.

\begin{figure}[htbp]
    \centering
    \includegraphics[width=\textwidth]{figures/fig5_specification_curve.pdf}
    \caption{Specification curve showing approximation quality across all 540 segment-level comparisons. Top panel: Wasserstein distances sorted from highest (worst) to lowest (best), with points colored by questionnaire type. Bottom panel: Indicator markers showing which factor levels (questionnaire, model, country, health status, income level) correspond to each specification. The dashed line indicates the overall mean (W = 2.09). Reverse Scale configurations (orange) cluster predominantly among the worst-performing specifications, while Netherlands and LLaMA 3.1 8B dominate the best-performing region.}
    \label{fig:specification-curve}
\end{figure}

\subsection{Variance Decomposition}
\label{subsec:variance-decomposition-results}

To quantify the relative importance of each factor in explaining variation in approximation quality, variance decomposition analysis was conducted across all 540 segment-level Wasserstein distances. Table~\ref{tab:variance-decomposition} presents the results.

\begin{table}[htbp]
\centering
\caption{Variance decomposition results: relative importance of factors in explaining Wasserstein distance variation}
\label{tab:variance-decomposition}
\begin{tabular}{lccc}
\hline
\textbf{Factor} & \textbf{Variance Explained (\%)} & \textbf{F} & \textbf{p} \\
\hline
Health Status & 11.0 & 33.27 & $<$.001 \\
Model & 8.0 & 23.27 & $<$.001 \\
Country & 7.0 & 13.48 & $<$.001 \\
Questionnaire Type & 6.9 & 9.86 & $<$.001 \\
Income Level & 0.7 & 1.78 & .170 \\
\hline
\end{tabular}
\end{table}

Health status emerged as the most important factor, explaining 11.0\% of variance in Wasserstein distances (F = 33.27, p $<$ .001). This finding has important implications for using synthetic data in health-related research, as discussed in Section~\ref{subsec:demographic-effects}.

Model choice explained 8.0\% of variance (F = 23.27, p $<$ .001), confirming that model selection meaningfully affects approximation quality. Country explained 7.0\% of variance (F = 13.48, p $<$ .001), while questionnaire type explained 6.9\% (F = 9.86, p $<$ .001).

Notably, income level showed minimal influence (0.7\% variance explained) and was not statistically significant (F = 1.78, p = .170). This null finding is substantively important: it suggests that LLMs approximate income-related differences in life satisfaction more uniformly than health-related differences, implying greater reliability for income-based comparisons in synthetic data.

Figure~\ref{fig:variance-decomposition} visualizes these results.

\begin{figure}[htbp]
    \centering
    \includegraphics[width=0.9\textwidth]{figures/fig2_variance_decomposition_forest.pdf}
    \caption{Variance decomposition results showing relative importance of each factor in explaining Wasserstein distance variation. Health status emerges as the dominant factor, while income level shows negligible importance and is not statistically significant.}
    \label{fig:variance-decomposition}
\end{figure}

\subsection{Demographic Effects}
\label{subsec:demographic-effects}

Given the strong influence of health status in the variance decomposition, detailed examination of demographic patterns is warranted. Table~\ref{tab:health-effects} presents Wasserstein distances by health status level.

\begin{table}[htbp]
\centering
\caption{Approximation quality by health status}
\label{tab:health-effects}
\begin{tabular}{lcccc}
\hline
\textbf{Health Status} & \textbf{Mean W} & \textbf{SD} & \textbf{Median} & \textbf{Range} \\
\hline
Good & 1.93 & 1.47 & 1.47 & 0.35--7.60 \\
Fair & 1.71 & 0.61 & 1.67 & 0.38--4.00 \\
Poor & 2.63 & 1.12 & 2.51 & 0.40--6.71 \\
\hline
\end{tabular}
\end{table}

Poor health populations showed substantially worse approximation (M = 2.63, SD = 1.12) compared to fair health (M = 1.71, SD = 0.61) and good health (M = 1.93, SD = 1.47) populations. The Wasserstein distance for poor health was 54\% higher than for fair health. Interestingly, fair health showed better approximation than good health, possibly because fair health represents a more ``average'' response pattern that LLMs can model more effectively.

In contrast, income level showed no meaningful effect on approximation quality (Table~\ref{tab:income-effects}). Mean Wasserstein distances ranged only from 1.97 (medium income) to 2.20 (low income), spanning just 0.23 units compared to 0.92 units for health status. This negligible variation confirms that LLMs represent income-satisfaction relationships more uniformly than health-satisfaction relationships.

\begin{table}[htbp]
\centering
\caption{Approximation quality by income level}
\label{tab:income-effects}
\begin{tabular}{lcccc}
\hline
\textbf{Income Level} & \textbf{Mean W} & \textbf{SD} & \textbf{Median} & \textbf{Range} \\
\hline
Low & 2.20 & 0.85 & 2.10 & 0.56--4.53 \\
Medium & 1.97 & 1.07 & 1.68 & 0.44--6.35 \\
High & 2.10 & 1.53 & 1.56 & 0.35--7.60 \\
\hline
\end{tabular}
\end{table}

\subsection{Model and Questionnaire Effects}
\label{subsec:model-questionnaire-effects}

The segment-level analysis confirms the country-level finding that smaller models outperform larger models (Table~\ref{tab:model-segment}). LLaMA 3.1 8B achieved the lowest mean Wasserstein distance (M = 1.63, SD = 0.73) and won in 83.3\% of segment-questionnaire combinations, producing the best approximation for 150 of 180 unique combinations. LLaMA 3.3 70B showed substantially worse performance (M = 2.24, SD = 1.32, 13.3\% win rate), while Qwen 2.5 72B performed worst (M = 2.41, SD = 1.27, 7.2\% win rate).

\begin{table}[htbp]
\centering
\caption{Model performance across 540 segment-level comparisons}
\label{tab:model-segment}
\begin{tabular}{lcccc}
\hline
\textbf{Model} & \textbf{Mean W} & \textbf{SD} & \textbf{Win Rate} & \textbf{Best Segment Count} \\
\hline
LLaMA 3.1 8B & 1.63 & 0.73 & 83.3\% & 150/180 \\
LLaMA 3.3 70B & 2.24 & 1.32 & 13.3\% & 24/180 \\
Qwen 2.5 72B & 2.41 & 1.27 & 7.2\% & 13/180 \\
\hline
\end{tabular}
\end{table}

Questionnaire patterns also held at the segment level (Table~\ref{tab:questionnaire-segment}). SWLS achieved the best mean Wasserstein distance (M = 1.78, SD = 0.71), followed by OHQ (M = 1.80, SD = 0.54). The multi-item scales showed lower variance than single-item scales, suggesting more consistent performance across demographic contexts. REVERSE again showed the worst performance (M = 2.63, SD = 1.69), with mean Wasserstein distance 48\% higher than SWLS, confirming that reversed scale processing represents a fundamental limitation for current LLMs.

\begin{table}[htbp]
\centering
\caption{Questionnaire performance across 540 segment-level comparisons}
\label{tab:questionnaire-segment}
\begin{tabular}{lcccc}
\hline
\textbf{Questionnaire} & \textbf{Mean W} & \textbf{SD} & \textbf{Median} & \textbf{Range} \\
\hline
SWLS & 1.78 & 0.71 & 1.63 & 0.60--4.71 \\
OHQ & 1.80 & 0.54 & 1.73 & 0.40--3.71 \\
Original WVS & 2.08 & 1.19 & 1.72 & 0.51--5.71 \\
CANTRIL & 2.16 & 1.24 & 1.82 & 0.35--6.71 \\
REVERSE & 2.63 & 1.69 & 2.02 & 0.60--7.60 \\
\hline
\end{tabular}
\end{table}

\subsection{Score-Level Accuracy}
\label{subsec:score-level-results}

The preceding subgroup analyses examined approximation quality across demographic segments using aggregate distributional metrics. This section investigates which specific life satisfaction scores (1--10) are predicted more accurately, revealing systematic patterns in LLM response generation. For each score $s \in \{1, \ldots, 10\}$, the absolute error is computed as the difference between real and synthetic proportions: $|p_s^{real} - p_s^{synth}|$. Aggregating across all 60 model-questionnaire-country combinations yields 600 score-level comparisons.

Table~\ref{tab:score-level-overall} presents prediction accuracy for each satisfaction level.

\begin{table}[htbp]
\centering
\caption{Score-level prediction accuracy (averaged across all conditions)}
\label{tab:score-level-overall}
\begin{tabular}{cccccc}
\hline
\textbf{Score} & \textbf{Real Prop.} & \textbf{Synth. Prop.} & \textbf{Mean Abs. Error} & \textbf{SD} & \textbf{Rank} \\
\hline
1 & 0.015 & 0.007 & 0.020 & 0.023 & 1 (Best) \\
2 & 0.011 & 0.045 & 0.046 & 0.044 & 3 \\
3 & 0.020 & 0.036 & 0.044 & 0.057 & 2 \\
4 & 0.031 & 0.065 & 0.065 & 0.060 & 4 \\
5 & 0.075 & 0.106 & 0.094 & 0.085 & 5 \\
6 & 0.082 & 0.163 & 0.108 & 0.074 & 6 \\
7 & 0.174 & 0.153 & 0.117 & 0.071 & 7 \\
8 & 0.252 & 0.252 & 0.121 & 0.101 & 8 \\
9 & 0.135 & 0.156 & 0.148 & 0.088 & 9 \\
10 & 0.205 & 0.016 & 0.194 & 0.139 & 10 (Worst) \\
\hline
\end{tabular}
\end{table}

The results reveal that prediction accuracy varies substantially across scores. Low satisfaction scores (1--3) achieved the best accuracy (M = 0.021--0.044), while high scores (9--10) showed the worst accuracy (M = 0.199--0.238). One-way ANOVA confirmed significant variation across scores ($F(9, 590) = 34.09$, $p < .001$).

The most striking finding is the dramatic under-prediction of score 10: real data shows 20.5\% of responses at maximum satisfaction, but synthetic data produces only 2.2\%---an 89\% under-representation. This pattern indicates a systematic central tendency bias, where LLMs avoid generating extreme responses.

\subsubsection{Prediction Bias Direction}

Figure~\ref{fig:score-level-direction} displays the direction and magnitude of prediction bias for each score.

\begin{figure}[htbp]
    \centering
    \includegraphics[width=0.9\textwidth]{figures/fig_score_level_direction.pdf}
    \caption{LLM prediction bias by life satisfaction score. Positive values (red) indicate over-prediction; negative values (green) indicate under-prediction. Error bars show standard deviation.}
    \label{fig:score-level-direction}
\end{figure}

LLMs systematically over-predict moderate scores (2--6, 8--9) while under-predicting extreme scores. Score 6 is over-predicted by 3.4 percentage points (real: 8.2\%, synthetic: 11.6\%), while score 10 is under-predicted by 18.3 percentage points. Score 8 shows moderate over-prediction (real: 25.1\%, synthetic: 29.2\%).

This pattern is consistent with a central tendency bias in LLM response generation: rather than producing the full range of human responses, LLMs concentrate probability mass toward the middle of the scale. Such bias may reflect training data characteristics, where extreme statements of complete satisfaction or dissatisfaction are less common in natural language corpora than moderate expressions.

\subsubsection{Variation by Questionnaire Type}

Appendix Table~\ref{tab:app-score-level-by-questionnaire} presents prediction accuracy broken down by questionnaire type. Multi-item scales (SWLS, OHQ) show better accuracy for low scores (1--4), likely because equipercentile equating distributes responses more evenly. Original WVS performs best for moderate scores (5--7). Critically, score 10 remains poorly predicted across all questionnaires (errors 0.162--0.205), indicating that maximum satisfaction is fundamentally difficult for LLMs to generate at human-like frequencies regardless of measurement instrument.

Figure~\ref{fig:score-level-real-vs-synth} provides a visual comparison of real and synthetic distributions across models.

\begin{figure}[htbp]
    \centering
    \includegraphics[width=\textwidth]{figures/fig_score_level_real_vs_synth.pdf}
    \caption{Comparison of real WVS and synthetic score distributions by model. All models show similar biases: under-prediction of score 10 and over-prediction of scores 4--6.}
    \label{fig:score-level-real-vs-synth}
\end{figure}

All three models exhibit similar score-level biases, with the rightward skew characteristic of real life satisfaction data (concentration at scores 7--10) not fully captured by synthetic responses. LLaMA 3.1 8B shows slightly tighter alignment, consistent with its superior performance on aggregate metrics.

Overall, the score-level analysis reveals that LLM approximation accuracy decreases monotonically from low to high satisfaction levels. The dramatic under-prediction of score 10 (89\% under-representation) represents the single largest source of distributional mismatch. This central tendency bias has practical implications: synthetic data will systematically underestimate the proportion of highly satisfied individuals, and research focusing on extreme satisfaction levels should treat synthetic data with particular caution.

%-----------------------------------------------------------
\section{Ensemble Approaches}
\label{sec:ensemble-results}

The ensemble analysis addresses RQ3 by testing whether combining results from multiple questionnaires improves alignment with real survey responses beyond what any single questionnaire achieves.

\subsection{Ensemble Construction}
\label{subsec:ensemble-construction}

Two ensemble strategies were evaluated. The KS-based optimal weighting approach assigned weights based on inverse KS statistics, giving higher weight to better-performing questionnaires; for each model, weights were calculated as $w_q = (1 - \overline{KS}_q) / \sum_{q'} (1 - \overline{KS}_{q'})$, where $\overline{KS}_q$ is the average KS statistic for questionnaire $q$ across countries. The equal-weight averaging approach assigned all four questionnaires equal weight (25\% each), treating all measurement approaches as equally valid. Both ensembles excluded the REVERSE questionnaire due to its consistently poor performance, combining only Original WVS, CANTRIL, SWLS, and OHQ.

\subsection{Ensemble Performance}
\label{subsec:ensemble-performance}

Table~\ref{tab:ensemble-results} compares ensemble performance against individual questionnaires.

\begin{table}[htbp]
\centering
\caption{Ensemble performance compared to individual questionnaires (averaged across models and countries)}
\label{tab:ensemble-results}
\begin{tabular}{lcc}
\hline
\textbf{Approach} & \textbf{Mean W} & \textbf{Mean KS} \\
\hline
\multicolumn{3}{l}{\textit{Individual Questionnaires}} \\
Original WVS & 0.95 & 0.36 \\
SWLS & 1.03 & 0.34 \\
CANTRIL & 1.24 & 0.39 \\
OHQ & 1.46 & 0.45 \\
REVERSE & 2.53 & 0.53 \\
\hline
\multicolumn{3}{l}{\textit{Ensemble Approaches}} \\
KS-Based Ensemble & 0.70 & 0.25 \\
Equal-Weight Average & 0.71 & 0.25 \\
\hline
\end{tabular}
\end{table}

Both ensemble approaches substantially outperformed all individual questionnaires. The KS-based ensemble achieved mean Wasserstein distance of 0.70, representing a 26\% improvement over the best individual questionnaire (Original WVS, W = 0.95) and a 72\% improvement over the worst included questionnaire (OHQ, W = 1.46). The equal-weight average achieved nearly identical performance (W = 0.71), suggesting that all four questionnaires contribute relatively equally to ensemble performance.

KS statistics showed similar patterns: both ensembles achieved mean KS = 0.25, compared to 0.34 for the best individual questionnaire (SWLS), representing a 26\% improvement.

Figure~\ref{fig:wasserstein-ensemble} displays Wasserstein distances for all approaches across countries and models. The ensemble (light gray) and average (dark gray) approaches consistently achieve lower Wasserstein distances than individual questionnaires across nearly all model-country combinations, with particularly notable improvements for the larger models (LLaMA 3.3 70B and Qwen 2.5 72B).

\begin{figure}[htbp]
    \centering
    \includegraphics[width=\textwidth]{figures/wasserstein_all_approaches.png}
    \caption{Wasserstein distance comparison across all approaches. Each panel shows one country, with models on the x-axis and questionnaire types distinguished by color. Ensemble (light gray) and Average (dark gray) approaches consistently achieve lower Wasserstein distances than individual questionnaires, demonstrating the benefit of combining multiple measurement instruments.}
    \label{fig:wasserstein-ensemble}
\end{figure}

Table~\ref{tab:ensemble-by-model} presents ensemble performance by model.

\begin{table}[htbp]
\centering
\caption{Ensemble performance by model (Wasserstein distance)}
\label{tab:ensemble-by-model}
\begin{tabular}{lccc}
\hline
\textbf{Model} & \textbf{KS Ensemble} & \textbf{Equal-Weight} & \textbf{Best Individual} \\
\hline
LLaMA 3.1 8B & 0.78 & 0.79 & 0.89 (Original WVS) \\
LLaMA 3.3 70B & 0.71 & 0.72 & 0.85 (Original WVS) \\
Qwen 2.5 72B & 0.61 & 0.63 & 0.90 (Original WVS) \\
\hline
\end{tabular}
\end{table}

All three models showed improvement with ensemble approaches, with Qwen 2.5 72B showing the largest gain (32\% improvement from 0.90 to 0.61). This suggests that ensemble methods may be particularly beneficial for models with weaker individual questionnaire performance, as they can leverage complementary strengths across measurement instruments.

Table~\ref{tab:ensemble-by-country} presents ensemble performance by country.

\begin{table}[htbp]
\centering
\caption{Ensemble performance by country (Wasserstein distance, averaged across models)}
\label{tab:ensemble-by-country}
\begin{tabular}{lccc}
\hline
\textbf{Country} & \textbf{KS Ensemble} & \textbf{Equal-Weight} & \textbf{Best Individual} \\
\hline
Netherlands & 0.44 & 0.45 & 0.48 (Original WVS) \\
USA & 0.65 & 0.66 & 0.88 (Original WVS) \\
Indonesia & 0.86 & 0.87 & 1.08 (Original WVS) \\
Mexico & 0.86 & 0.87 & 0.88 (SWLS) \\
\hline
\end{tabular}
\end{table}

The ensemble showed improvements across all countries, with the largest gains in the USA (26\% improvement) and Indonesia (20\% improvement). The Netherlands, which already showed good individual questionnaire performance, showed more modest improvement (8\%), while Mexico showed minimal additional benefit from the ensemble approach.

Figures~\ref{fig:scatter-ensemble-usa}--\ref{fig:scatter-ensemble-mex} display the life satisfaction distributions for each country, comparing individual questionnaires with ensemble approaches. The ensemble and average distributions (gray lines) consistently track closer to the real WVS distribution (black line) than most individual questionnaires, particularly avoiding the extreme peaks exhibited by some individual approaches.

\begin{figure}[htbp]
    \centering
    \includegraphics[width=\textwidth]{figures/scatter_all_approaches_USA.png}
    \caption{Life satisfaction distributions for the United States comparing all approaches. Black line shows real WVS data; colored lines show individual questionnaires; gray lines show ensemble (light) and average (dark) approaches. Each panel represents one LLM.}
    \label{fig:scatter-ensemble-usa}
\end{figure}

\begin{figure}[htbp]
    \centering
    \includegraphics[width=\textwidth]{figures/scatter_all_approaches_NLD.png}
    \caption{Life satisfaction distributions for the Netherlands comparing all approaches. The Netherlands shows the best overall alignment, with ensemble approaches achieving near-perfect approximation.}
    \label{fig:scatter-ensemble-nld}
\end{figure}

\begin{figure}[htbp]
    \centering
    \includegraphics[width=\textwidth]{figures/scatter_all_approaches_IDN.png}
    \caption{Life satisfaction distributions for Indonesia comparing all approaches. The ensemble approaches better capture the right-skewed distribution characteristic of Indonesian responses.}
    \label{fig:scatter-ensemble-idn}
\end{figure}

\begin{figure}[htbp]
    \centering
    \includegraphics[width=\textwidth]{figures/scatter_all_approaches_MEX.png}
    \caption{Life satisfaction distributions for Mexico comparing all approaches. Mexico shows the highest variation across approaches, with ensemble methods providing moderate improvement.}
    \label{fig:scatter-ensemble-mex}
\end{figure}

\subsection{Practical Implications for RQ3}
\label{subsec:rq3-implications}

These results provide a clear affirmative answer to RQ3: combining results across multiple scales does improve alignment with human survey responses. The ensemble approach achieved approximately 26\% better Wasserstein distance and 26\% better KS statistic compared to the best individual questionnaire.

The near-identical performance of KS-based weighting (W = 0.70) and equal-weight averaging (W = 0.71) has practical implications. Given that the performance difference is minimal (1.4\%), the equal-weight approach is recommended for most applications due to its simplicity, transparency, and robustness. The small performance gap suggests that all four questionnaires (Original WVS, CANTRIL, SWLS, OHQ) contribute roughly equally to ensemble performance when REVERSE is excluded.

%-----------------------------------------------------------
\section{Summary of Findings}
\label{sec:results-summary}

This chapter presented findings from three complementary analytical approaches examining whether LLMs can approximate human life satisfaction survey responses. Table~\ref{tab:findings-summary} summarizes the key findings organized by research question.

\begin{table}[htbp]
\centering
\caption{Summary of key findings by research question}
\label{tab:findings-summary}
\begin{tabular}{p{3.5cm}p{9cm}}
\hline
\textbf{Research Question} & \textbf{Key Findings} \\
\hline
\textbf{RQ1}: Can LLMs approximate human responses? &
Partially. Mean Wasserstein distance (W = 1.44) indicates moderate approximation quality, though performance varies substantially across conditions (range: 0.16--3.28). LLaMA 3.1 8B outperforms larger models (83.3\% win rate). Poor health populations are systematically harder to approximate (54\% higher W than fair health). \\
\hline
\textbf{RQ2}: Are certain scales better approximated? &
Yes. SWLS and Original WVS show best performance (W $\approx$ 1.0). REVERSE performs substantially worse (W = 2.53, 166\% higher than Original WVS). Questionnaire type explains 6.9\% of variance in approximation quality. \\
\hline
\textbf{RQ3}: Can ensembles improve alignment? &
Yes. Both KS-based and equal-weight ensembles achieve W $\approx$ 0.70, representing 26\% improvement over best individual questionnaire. Equal-weight averaging is recommended for simplicity. \\
\hline
\end{tabular}
\end{table}

The variance decomposition analysis revealed that health status (11.0\%), model choice (8.0\%), country (7.0\%), and questionnaire type (6.9\%) are the significant factors influencing approximation quality, while income level has negligible impact (0.7\%). This finding has important implications for using synthetic survey data: health-related research requires additional validation, while income-based analyses may be more reliable.

The score-level analysis revealed a systematic central tendency bias: LLMs dramatically under-predict maximum satisfaction (score 10 is under-represented by 89\%) while over-predicting moderate scores (2--6, 8--9). This pattern persists across all models and questionnaires, representing the single largest source of distributional mismatch and indicating that research focusing on extreme satisfaction levels should treat synthetic data with particular caution.

The counterintuitive finding that smaller models outperform larger models challenges prevailing assumptions and has practical implications for cost-effective synthetic data generation. The consistent success of ensemble approaches across models and countries suggests a robust strategy for improving synthetic data quality when multiple measurement instruments are feasible.

%-----------------------------------------------------------
% End of Chapter 5

% Chapter 6: Limitations

\chapter{Limitations}
\label{ch:limitations}

Several limitations of the research design, data, and analytical approach should be considered when interpreting the findings.

The study examined four countries (USA, Indonesia, Netherlands, Mexico), selected specifically for their non-normal life satisfaction distributions. While this selection enabled rigorous testing of whether LLMs can capture complex distributional shapes, it limits generalizability to countries with different response patterns. Countries with approximately normal distributions, excluded by design, may show different approximation dynamics. Relatedly, the study implicitly assumed that normally distributed responses might represent a ``default'' pattern that LLMs could approximate through random or uninformed guessing. However, it is equally plausible that LLMs have learned to produce skewed distributions as their default for life satisfaction questions, reflecting the response patterns prevalent in Western, Educated, Industrialized, Rich, and Democratic (WEIRD) populations that dominate their training corpora. This interpretation aligns with the observed pattern where the Netherlands showed the best approximation quality while Mexico showed the worst---the LLMs may be reproducing culturally specific response patterns rather than learning to match any arbitrary distribution. Within these countries, the demographic segmentation was limited to income level and health status, based on feature importance analysis identifying these as the strongest predictors of life satisfaction. Other demographic factors (age, gender, education, marital status, employment) were not systematically varied, and the findings regarding demographic effects may not extend to these unmeasured dimensions. Each demographic segment received 30 synthetic responses, chosen to balance computational costs with statistical power; larger samples might reveal finer-grained patterns or reduce sampling variability.

Beyond sample scope, model selection introduces constraints. Three LLMs were evaluated: LLaMA 3.1 8B, LLaMA 3.3 70B, and Qwen 2.5 72B, all open-weights models accessed through an academic cloud endpoint. Proprietary models (GPT-4, Claude) were not tested due to access constraints and cost considerations, and the finding that smaller models outperform larger ones may not generalize to proprietary architectures with different training procedures. Additionally, the models were queried with default temperature settings; systematic variation of generation parameters might yield different response distributions, but this parameter space was not explored. The study also employed a single prompting strategy (interview-based prompting), selected based on prior evidence of its effectiveness \parencite{lutz2025prompt}. Alternative strategies were tested during pilot work but not systematically compared in the main analysis, so the findings are conditional on this specific prompting approach.

Temporal factors present another source of uncertainty. The real survey data comes from World Values Survey Wave 7 (2017--2022), while the LLMs were trained on text corpora with different temporal coverage. This mismatch means that the models may have learned response patterns from earlier periods that do not reflect current population attitudes. Significant events such as the COVID-19 pandemic occurred during the WVS data collection period, potentially affecting life satisfaction distributions in ways not captured by the LLMs. Compounding this issue, the prompts did not specify any time frame for the synthetic respondents, leaving ambiguous whether generated responses reflect contemporary attitudes, historical patterns from the training data, or some mixture of both.

Finally, measurement and validation choices affect interpretation. The equipercentile equating procedure used to transform SWLS and OHQ scores to a common 1-10 scale involves assumptions about score correspondence that may introduce artifacts affecting comparisons across questionnaire types. More fundamentally, the evaluation focused on distributional alignment using Wasserstein distance and KS statistics, metrics that capture overall distributional similarity but do not assess whether synthetic responses are valid at the individual level. An LLM might produce plausible aggregate distributions through mechanisms quite different from human response processes. The study also treated WVS responses as ground truth, though survey responses themselves are subject to measurement error, social desirability bias, and other validity threats. Moreover, the same WVS participants might have responded differently had they been administered the alternative scales (SWLS, OHQ, Cantril); the comparison thus evaluates alignment with one particular operationalization of life satisfaction rather than with participants' ``true'' satisfaction levels across measurement approaches. The evaluation therefore assesses alignment with measured responses rather than with ``true'' underlying life satisfaction levels. Whether the patterns observed, particularly the advantages of certain questionnaire formats and the challenges with reversed scales, extend to other survey constructs such as political attitudes, health behaviors, or consumer preferences remains untested.

% Chapter 7: Conclusion

\chapter{Conclusion}
\label{ch:conclusion}

This thesis investigated whether large language models can generate synthetic survey responses that approximate real human life satisfaction distributions. Through systematic comparison of LLM-generated responses with World Values Survey data across four countries, five questionnaire formats, and three models, the research addressed three questions about approximation quality, questionnaire design effects, and ensemble improvement strategies.

Regarding the first research question, to what extent LLMs can approximate human responses, the findings indicate moderate approximation with substantial variation across conditions. The best configurations achieved near-perfect alignment (Wasserstein distance of 0.16 for the Netherlands with LLaMA 3.1 8B), while poor configurations showed large discrepancies (W > 3.0). Mean performance across all 540 segment-level comparisons was W = 2.09. A counterintuitive finding emerged: the smallest model (LLaMA 3.1 8B) consistently outperformed larger models, achieving an 83\% win rate across segment-questionnaire combinations. Health status emerged as the most influential demographic factor, explaining 11\% of variance in approximation quality, while income level showed negligible influence.

The second research question, whether certain scales are better approximated, received a clear affirmative answer. The Original WVS and SWLS formats achieved the best performance (mean W $\approx$ 1.0), while the Reverse Scale performed substantially worse (mean W = 2.53), 166\% higher than the Original WVS. This indicates that LLMs struggle to correctly process reversed scale orientations. Multi-item scales showed more consistent performance across demographic contexts than single-item measures.

The third research question, whether ensemble approaches can improve alignment, also received strong support. Both KS-based optimal weighting and equal-weight averaging achieved mean Wasserstein distances of approximately 0.70, a 26\% improvement over the best individual questionnaire. The near-identical performance of both weighting schemes suggests that equal-weight averaging is preferable for practical applications due to its simplicity.

These findings carry several implications for researchers considering LLM-generated survey data. Synthetic data can achieve reasonable distributional alignment under favorable conditions, but quality varies substantially across demographic groups, questionnaire formats, and target populations. Model selection matters, but not in the expected direction: smaller models may outperform larger ones, suggesting computational resources could be better allocated to generating more responses rather than using larger architectures. Questionnaire design requires careful consideration: reversed scales should be avoided, and combining responses across multiple formats through ensemble methods provides measurable improvement. Certain demographic contexts pose greater challenges, with health status influencing approximation quality more than income.

Several directions for future research emerge from this work. Extending the evaluation to other survey constructs would test whether these patterns generalize beyond life satisfaction. Systematic comparison of prompting strategies could identify approaches that further improve approximation quality. Evaluation of proprietary models would clarify whether the smaller-is-better finding reflects general principles or specific characteristics of open-weights architectures. More fundamentally, the mechanisms underlying LLM survey simulation remain unclear; why smaller models outperform larger ones and why reversed scales pose such difficulty are questions that could inform both model development and survey design. Finally, practical deployment of synthetic survey data requires validation frameworks and guidelines for when synthetic data is sufficient for specific research purposes.


\cleardoublepage
\printbibliography[heading=bibintoc]

% Appendix
% This appendix contains supplementary figures and tables

\appendix

\chapter{Supplementary Results}
\label{app:supplementary}

\section{Kolmogorov-Smirnov Statistics}
\label{app:ks-results}

This section presents Kolmogorov-Smirnov (KS) statistics as a complementary measure to the Wasserstein distances reported in the main text. While Wasserstein distance captures the total amount of probability mass movement required to transform one distribution into another, the KS statistic measures the maximum vertical distance between cumulative distribution functions. Both metrics are reported throughout this thesis to provide a comprehensive assessment of distributional alignment.

Figure~\ref{fig:ks-by-country} presents KS statistics for all model-questionnaire combinations across the four countries. The patterns observed mirror those found with Wasserstein distance: REVERSE consistently shows the highest KS values, Original WVS and SWLS achieve the best performance, and LLaMA 3.1 8B generally outperforms larger models.

\begin{figure}[htbp]
    \centering
    \includegraphics[width=\textwidth]{figures/fig_ks_by_country.png}
    \caption{Kolmogorov-Smirnov statistic by model and questionnaire across countries. Each panel represents one country, with models grouped on the x-axis and questionnaire types distinguished by color. Lower bars indicate better approximation. The KS statistic measures the maximum vertical distance between the cumulative distribution functions of real and synthetic data.}
    \label{fig:ks-by-country}
\end{figure}

Table~\ref{tab:ks-summary} provides summary statistics for KS values across all 60 country-level comparisons.

\begin{table}[htbp]
\centering
\caption{Summary of Kolmogorov-Smirnov statistics across all country-level comparisons}
\label{tab:ks-summary}
\begin{tabular}{lcccc}
\hline
\textbf{Grouping} & \textbf{Category} & \textbf{Mean KS} & \textbf{SD} & \textbf{Range} \\
\hline
\multirow{3}{*}{By Model}
    & LLaMA 3.1 8B & 0.39 & 0.12 & 0.18--0.62 \\
    & LLaMA 3.3 70B & 0.40 & 0.14 & 0.19--0.68 \\
    & Qwen 2.5 72B & 0.46 & 0.13 & 0.24--0.71 \\
\hline
\multirow{5}{*}{By Questionnaire}
    & Original WVS & 0.36 & 0.11 & 0.18--0.54 \\
    & SWLS & 0.34 & 0.09 & 0.21--0.52 \\
    & CANTRIL & 0.39 & 0.11 & 0.22--0.58 \\
    & OHQ & 0.45 & 0.10 & 0.28--0.63 \\
    & REVERSE & 0.53 & 0.12 & 0.32--0.71 \\
\hline
\multirow{4}{*}{By Country}
    & Netherlands & 0.38 & 0.14 & 0.18--0.68 \\
    & USA & 0.40 & 0.11 & 0.22--0.59 \\
    & Indonesia & 0.40 & 0.12 & 0.21--0.62 \\
    & Mexico & 0.47 & 0.13 & 0.24--0.71 \\
\hline
\end{tabular}
\end{table}

The KS statistics confirm the main findings: (1) the smallest model (LLaMA 3.1 8B) achieves the best average KS statistic; (2) SWLS and Original WVS show the best questionnaire performance; (3) REVERSE shows substantially worse performance than other questionnaires; and (4) Mexico presents the greatest approximation challenge while the Netherlands shows the best alignment.

\section{Supplementary Figures}
\label{app:supplementary-figures}

This section presents additional visualizations that complement the main results.

\begin{figure}[htbp]
    \centering
    \includegraphics[width=\textwidth]{figures/fig4_health_impact.pdf}
    \caption{Approximation quality by health status across countries. Each panel shows one country, with questionnaire types on the x-axis and Wasserstein distance on the y-axis. Poor health populations (darkest bars) consistently show the highest Wasserstein distances across all countries and most questionnaire types, demonstrating that the health effect is robust across geographic and measurement contexts.}
    \label{fig:health-impact}
\end{figure}

\begin{figure}[htbp]
    \centering
    \includegraphics[width=\textwidth]{figures/ks_all_approaches.png}
    \caption{Kolmogorov-Smirnov statistic comparison across all approaches including ensemble methods. Each panel shows one country, with models on the x-axis and questionnaire types distinguished by color. Ensemble (light gray) and Average (dark gray) approaches consistently achieve lower KS statistics than individual questionnaires, confirming the Wasserstein distance findings.}
    \label{fig:ks-ensemble}
\end{figure}

\section{Score-Level Analysis by Questionnaire Type}
\label{app:score-level-questionnaire}

Table~\ref{tab:app-score-level-by-questionnaire} presents prediction accuracy broken down by questionnaire type at the individual score level.

\begin{table}[htbp]
\centering
\caption{Score-level prediction accuracy by questionnaire type (mean absolute error)}
\label{tab:app-score-level-by-questionnaire}
\begin{tabular}{c|ccccc}
\hline
\textbf{Score} & \textbf{Original WVS} & \textbf{Cantril} & \textbf{Reverse} & \textbf{SWLS} & \textbf{OHQ} \\
\hline
1 & 0.015 & 0.015 & 0.042 & 0.015 & 0.015 \\
2 & 0.070 & 0.070 & 0.068 & 0.011 & 0.011 \\
3 & 0.060 & 0.080 & 0.042 & 0.020 & 0.020 \\
4 & 0.138 & 0.095 & 0.032 & 0.031 & 0.031 \\
5 & 0.058 & 0.098 & 0.060 & 0.179 & 0.075 \\
6 & 0.048 & 0.123 & 0.106 & 0.096 & 0.168 \\
7 & 0.106 & 0.102 & 0.137 & 0.125 & 0.115 \\
8 & 0.153 & 0.098 & 0.198 & 0.082 & 0.075 \\
9 & 0.102 & 0.135 & 0.108 & 0.126 & 0.271 \\
10 & 0.205 & 0.205 & 0.203 & 0.197 & 0.162 \\
\hline
\end{tabular}
\end{table}

Multi-item scales (SWLS, OHQ) show better accuracy for low scores (1--4), likely because equipercentile equating distributes responses more evenly. Original WVS performs best for moderate scores (5--7). Score 10 remains poorly predicted across all questionnaires (errors 0.162--0.205), indicating that maximum satisfaction is fundamentally difficult for LLMs to generate at human-like frequencies regardless of measurement instrument.


% Use of Generative AI Tools

\chapter{Use of Generative AI Tools}
\label{ch:genai}

As part of the research methodology, I used LLaMA 3.1 8B, LLaMA 3.3 70B, and Qwen 2.5 72B to generate synthetic survey responses. These models constitute the subject of the empirical investigation and are described in detail in Chapter~\ref{ch:method}.

In addition, I used the following generative AI tools for thesis preparation in accordance with the guidelines of the University of Mannheim:

\begin{itemize}
    \item \textbf{Claude (Anthropic):} Used for code generation in Python, formatting of LaTeX source code, generating descriptive comments for Python scripts, structuring the Git repository, and paraphrasing complex paragraphs.
    \item \textbf{Grammarly:} Used for proofreading and checking grammatical errors in the written text.
    \item \textbf{DeepL:} Used for English grammar and wording choice.
\end{itemize}

All AI-assisted content was reviewed, verified, and adapted by me. The intellectual contribution, research design, analysis, and conclusions presented in this thesis are entirely my own work.

\chapter{Declaration}
\label{ch:declaration}

I assert that this paper was written by me personally and that I was not assisted in any way by someone else. Furthermore, I assert that this paper or parts thereof have not been submitted elsewhere, neither by me nor by others. When I consulted print or electronic sources and publications to draw upon the writings or thoughts of others, I cited these sources. All secondary literature and additional sources have been acknowledged and are listed in the bibliography. The same is true for graphs, pictures, and all internet sources. Moreover, I consent that my paper may be screened and saved electronically and anonymously in order to be checked for plagiarism. I am aware that my paper may not be graded if I refuse to agree to these conditions.

\vspace{2cm}

\noindent
\begin{tabular}{@{}p{6cm}p{6cm}@{}}
\rule{5cm}{0.4pt} & \rule{5cm}{0.4pt} \\
Milana Taova & Place, Date
\end{tabular}


\end{document}
